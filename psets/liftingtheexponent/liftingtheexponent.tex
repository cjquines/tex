\documentclass[11pt,paper=letter]{scrartcl}
\usepackage[alttitle]{cjquines}

\begin{document}

\title{Lifting the exponent}

\author{Carl Joshua Quines}

\date{July 20, 2019}

\maketitle

\subsubsection*{Valuation}

Define $\nu_p(n)$ for positive integers $n$ as
$$\nu_p(n) = k \quad \iff \quad p^k \mid n,\,p^{k+1}\nmid n.$$
This is known as the $p$-adic valuation of $n$. Note that this is $\nu_p$ with the Greek letter $\nu$ (spelled nu, pronounced ``new'').\footnote{This is sometimes written as $v_p$ with the English letter $v$. I don't think this is standard, as I see more sources use $\nu_p$. I don't even know why $\nu$ is the letter chosen for this, other than its superficial similarity to the letter $v$.} Some properties that you should convince yourself are true:
\begin{itemize}
  \item $\nu_p(ab) = \nu_p(a) + \nu_p(b)$.
  \item Similarly $\nu_p\del{\frac ab} = \nu_p(a) - \nu_p(b)$. We can use this to extend the definition of $\nu_p$ to be a function from $\QQ^{\neq 0} \to \ZZ$.

  What should $\nu_p(0)$ be? To satisfy the product rule, we can pick $\nu_p(0) = \infty$.

  \item $\nu_p(a + b) \ge \min\cbr{\nu_p(a), \nu_p(b)}$, equality holds if $\nu_p(a) \neq \nu_p(b)$.
  \item $\nu_p(a - b) \ge e \iff a \equiv b \pmod{p^e}$ from the previous.
  \item $\nu_p(\gcd(a, b)) = \min\cbr{\nu_p(a), \nu_p(b)}$.
  \item $\nu_p(\lcm(a, b)) = \max\cbr{\nu_p(a), \nu_p(b)}$.
  \item $a = b \iff \nu_p(a) = \nu_p(b)$ for all $p$.
  \item More generally, $a \mid b \iff \nu_p(a) \le \nu_p(b)$ for all $p$.
\end{itemize}

\subsubsection*{Examples}

\begin{enumerate}
  \item Prove that $\displaystyle \gcd(a, b, c) = \frac{abc\cdot\lcm(a, b, c)}{\lcm(a, b)\lcm(b, c)\lcm(c, a)}$.

  \textbf{Sketch:} Pick a prime $p$, idea is to show $\nu_p$ of LHS and RHS are the same. Let $x = \nu_p(a)$, $y = \nu_p(b)$, and $z = \nu_p(c)$. In the LHS you have $\min\cbr{x, y, z}$, on the RHS you have $x + y + z + \max\cbr{x, y, z} - \max\cbr{x, y} - \max\cbr{y, z} - \max\cbr{z, x}$. But these are equal.

  \item Suppose $a \mid b^2 \mid a^3 \mid b^4 \mid a^5 \mid \cdots$. Prove that $a = b$.

  \textbf{Sketch:} This is an easy problem, but it's a bit hard to write up. Using $\nu_p$ makes it easier. We have
  $$a^{2n - 1} \mid b^{2n} \implies (2n-1)\nu_p(a) \le 2n\nu_p(b) \implies \nu_p(a) \le \frac{2n}{2n-1} \nu_p(b).$$
  Taking the limit as $n \to \infty$ means $\nu_p(a) \le \nu_p(b)$; similarly we can prove $\nu_p(b) \le \nu_p(a)$. This shows $\nu_p(a) = \nu_p(b)$.

  \item Let $p$ prime, $n \in \NN$. Suppose $p \mid \mid 2^n - 1$. Show that $p \mid \mid 2^{p-1} - 1$. (We say $p \mid \mid n \iff p \mid n, p^2 \nmid n$.)

  \textbf{Remark:} While this is typically done with the so-called \emph{lifting the exponent lemma}, many people learn the statement without knowing the proof, which I think is bad, because the proof gives useful intuition. So we're going to motivate the proof using this problem and the next problem.

  \textbf{Sketch:} Let $m$ be the order of $2$ modulo $p$. That is, the smallest positive integer $m$ such that $p \mid 2^m - 1$. Because $m$ is the order, we have $m \mid n$, so $2^m - 1 \mid 2^n - 1$, therefore, we get $p \mid \mid 2^m - 1$.

  Now we use the main idea, and that's dividing $2^{p-1} - 1$ by $2^m - 1$. With some algebra,
  $$\frac{2^{p-1} - 1}{2^m - 1} = 1 + 2^m + 2^{2m} + \cdots + 2^{p-1 - m}.$$
  Modulo $p$, this is $\frac{p- 1}{m}$ (because $p \mid 2^m - 1$). So this is not equal to $0$, so $p^2 \nmid 2^{p-1} - 1$. But by FLT, $p \mid 2^{p-1} - 1$, the conclusion follows.

  \item Let $n \in \NN^0$. Find $\nu_3\del{2^{3^n} + 1}$.

  \textbf{Sketch:} This is induction. Find the answer when $n = 0$. Then observe that
  $$\frac{2^{3^{n+1}} + 1}{2^{3^n} + 1} = 2^{2\cdot 3^n} - 2^{3^n} + 1 \equiv 1 - (-1) + 1 \equiv 3 \pmod 9,$$
  then it's divisible by $3$ but not $9$, so going $n \to n + 1$ increases $\nu_3$ by $1$.
\end{enumerate}

\subsubsection*{Lifting the exponent}

We can now state and prove the lifting the exponent lemma. It states that if $p$ is an odd prime, $p \nmid a$, $p \nmid b$, and $p \mid a - b$, then
$$\nu_p(a^n - b^n) = \nu_p(a - b) + \nu_p(n)$$
for all positive integers $n$. \textbf{The condition $p \mid a - b$ is very important, yet easy to forget. Always remember to check this condition.} In particular, you must have $\nu_p(a - b) > 0$.

The proof is by induction on $n$. The main idea here is the inductive step. The idea is that we want to take out the powers of $p$ from $n$. For example, if we take $n = p^{\alpha}$, we can rewrite this as
$$\nu_p\del{ \del{a^{p^{\alpha-1}}}^p - \del{b^{p^{\alpha-1}}}^p } = \nu_p\del{a^{p^{\alpha-1}} - b^{p^{\alpha-1}}} + 1.$$
But to prove this, we only have to show that it's true for $n = p$. Similarly, if we have $n = p^{\alpha}\beta$, where $\gcd(p, \beta) = 1$, we can write
$$\nu_p\del{\del{a^{p^{\alpha}}}^\beta - \del{b^{p^{\alpha}}}^\beta} = \nu_p\del{a^{p^\alpha} - b^{p^\alpha}},$$
which means we only have to show the case when $\nu_p(n) = 0$. This is already our inductive step! So these two cases, the one where $\nu_p(n) = 0$ and $n = p$, will form the two base cases of our induction.

The case $\nu_p(n) = 0$ is easy. Write
$$\nu_p(a^n - b^n) = \nu_p(a - b) \quad \iff \quad \nu_p\del{\frac{a^n - b^n}{a - b}} = 0;$$
where we get the second equation by transposing $\nu_p(a - b)$ and applying the quotient rule. We only need to show that 
$$p \nmid a^{n-1} + a^{n-2}b + \cdots + b^{n-1}.$$
This follows because $a \equiv b \pmod p$, so substitute this to get $$a^{n-1} + a^{n-1} + \cdots + a^{n-1} \equiv n a^{n-1} \not\equiv 0.$$
The other base case, $n = p$, is harder. We need to show that
$$\nu_p(a^p - b^p) = \nu_p(a - b) + 1 \quad \iff \quad \nu_p\del{\frac{a^p - b^p}{a - b}} = 1.$$
There are two parts here. First, we want to show
$$p \mid a^{p-1} + a^{p-2}b + \cdots + b^{p-1}.$$
This follows because $a \equiv b \pmod p$, so using a similar process from the other base case, we get $pa^{p-1} \equiv 0$. Second, we want to show that
$$p^2 \nmid a^{p-1} + a^{p-2}b + \cdots + b^{p-1}.$$
This second part is an algebra bash. We substitute $b \equiv pk + a \pmod{p^2}$, then expand with the binomial theorem. It's not that bad because all of the terms with $p^2$ disappear, leaving us with
$$a^{p-1} + \del{a^{p-1} + a^{p-2}pk} + \del{a^{p-1} + 2a^{p-2}pk} + \cdots + \del{a^{p-1} + (p-1)a^{p-2}pk}.$$
The $a^{p-2}pk$ terms have coefficients $1 + 2 + \cdots + p-1 \equiv 0 \pmod p$, so coupled with the extra $p$ factor, they all sum to $0 \pmod p^2$. This leaves you with $pa^{p-1} \not\equiv 0 \pmod{p^2}$.

\vspace{1em}
\noindent An alternative formulation follows if $n$ is odd. Then we can replace $b$ with $-b$ to get
$$\nu_p(a^n + b^n) = \nu_p(a + b) + \nu_p(n).$$
Note, again, this only applies if $n$ is odd.

\vspace{1em}
\noindent \textbf{Example:} Suppose $a, b, n, p, k \in \NN$ such that $n > 1$ is odd, $p$ is an odd prime, and $a^n + b^n = p^k$. Prove that $n$ is a power of $p$.

\vspace{1em}
\noindent \textbf{Sketch:} \textbf{Check all the conditions before using LTE!} We have $p$ is an odd prime. If $p \mid a$, then $p \mid b$, and we can divide both $a$ and $b$ by $p$ until neither is divisible by $p$, so WLOG $p \nmid a$ and $p \nmid b$. Also, $n$ is odd so we can use the $+$ case of LTE.

Now we check the hard condition. By factorization, since $a + b \mid a^n + b^n = p^k$, it must follow that either $a + b = 1$ (impossible) or $p \mid a + b$. This gives us all the conditions and now we can use LTE:
$$k = \nu_p\del{p^k} = \nu_p\del{a^n + b^n} = \nu_p(a + b) + \nu_p(n).$$
Now suppose $\ell = p^{\nu_p(n)}$. Then
$$\nu_p\del{a^{\ell} + b^{\ell}} = \nu_p(a + b) + \nu_p(n).$$
So $p^k \mid a^{\ell} + b^{\ell} \mid a^n + b^n = p^k$, so they must all be equal and $n = \ell$ which is a power of $p$.

\subsubsection*{Problems}

\begin{enumerate}
  \item (Folklore) Fix $k \in \NN$. Find all $n$ such that $3^k \mid 2^n - 1$.

  \item (Iran 2008) Fix $a \in \NN$. Suppose $4(a^n + 1)$ is a perfect cube for all $n \in \NN$. Prove that $a = 1$.

  \item (Ireland 1996) If $2^p + 3^p = a^n$ for some prime $p$, prove $n = 1$.

  \item (ISL 1991) Find the largest $k$ such that $1991^k \mid 1990^{1991^{1992}} + 1992^{1991^{1990}}$.

  \item (AIME 2018) Find the smallest $n$ such that $3^n$ ends with $01$ when written in base $143$.
\end{enumerate}

\subsubsection*{Hints}

\begin{enumerate}
  \item $2^{2n} - 1 = 4^n - 1$ and $3 \mid 4 - 1$.
  \item Taking $a^2 + 1$ mod $4$, we see it's never a power of $2$.
  \item $2^p + 3^p$ is not a square. Find $\nu_5\del{2^p + 3^p}$.
  \item $1990^{1991^{1992}} = \del{1990^{1991^2}}^{1991^{1990}}$.
  \item $11 \mid 3^5 - 1$ so $3^n - 1 = (3^5)^{n/5} - 1$.
\end{enumerate}

\subsubsection*{References}

The classic reference is Amir Hossein Parvardi's \href{https://www.imosuisse.ch/smo/skripte/unused/Lifting_the_exponent_EN.pdf}{Lifting the Exponent Lemma} handout, but I don't think it motivates LTE well enough. The exposition here roughly follows Evan Chen's \href{http://web.evanchen.cc/textbooks/OTIS-Excerpts.pdf}{OTIS Excerpts}.

Thanks to Konwoo Kim for sending a correction. 

\end{document}
