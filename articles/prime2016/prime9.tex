\documentclass[10pt,paper=letter]{scrartcl}
\usepackage[alttitle]{cjquines}

\begin{document}

\title{VCSMS PRIME}
\subtitle{Session 9: Geometry 2}
\author{compiled by Carl Joshua Quines}
\date{October 19, 2016}

\maketitle

\subsubsection*{Ad hoc}

\begin{enumerate}

\item (16QIII1) In the right triangle $ABC$, where $\angle B = 90\dg, BC : AB = 1 : 2$, construct the median $BD$ and let point $E$ be on $BD$ such that $CE \perp BD$. Determine $BE : ED$.

\item (14NA9) A circle with diameter $2$ is tangent to both diagonals of a square with side length of $2$. The circle intersects the square at points $P$ and $Q$. Find the length of segment $PQ$.

\item (9N5) Segments $AC$ and $BD$ intersect at point $P$ such that $PA = PD$ and $PB = PC$. Let $E$ be the foot of the perpendicular from $P$ to the line $CD$. Prove that the line $PE$ and the perpendicular bisectors of $PA$ and $PB$ are concurrent.

\item (10N2) On a cyclic quadrilateral $ABCD$, there is a point $P$ on side $AD$ such that the triangle $CDP$ and the quadrilateral $ABCP$ have equal perimeters and equal areas. Prove that two sides of $ABCD$ have equal lengths.

\item (8N3) Let $P$ be a point outside a circle, and let the two tangent lines through $P$ touch the circle at $A$ and $B$. Let $C$ be a point on the minor arc $AB$, and let ray $PC$ intersect the circle again at another point $D$. Let $L$ be the line that passes through $B$ parallel to $PA$, and let let $L$ intersect rays $AC$ and $AD$ at points $E$ and $F$, respectively. Prove that $B$ is the midpoint of $EF$.

\end{enumerate}

\subsubsection*{Triangles}

\begin{enumerate}

\item (15AI5) Triangle $ABC$ has a right angle at $B,$ with $AB = 3$ and $BC = 4$. If $D$ and $E$ are points on $AC$ and $BC$, respectively, such that $CD = DE = \dfrac{5}{3}$, find the perimeter of quadrilateral $ABED$.

\item (16AI11) Circle $O$ is inscribed in the right triangle $ACE$ with $\angle ACE = 90\dg$, touching sides $AC, CE$ and $AE$ at points $B, D$ and $F$, respectively. The length of $AB$ is twice the length of $BC$. Find the length of $CE$ if the perimeter of $ACE$ is $36$ units.

\item (8AII2) Let $ABC$ be an acute-angled triangle. Let $D$ and $E$ be points on $BC$ and $AC$ such that $AD \perp BC$ and $BE \perp AC$. Let $P$ be the point where ray $AD$ meets the semicircle constructed outwardly on $BC$, and $Q$ be the point where ray $BE$ meets the semicircle constructed outwardly on $AC$. Prove that $PC = QC$.

\item (9AII3) The bisector of $\angle BAC$ intersects the circumcircle of triangle $ABC$ again at $D$. Let $AD$ and $BC$ intersect at $E$, and $F$ be the midpoint of $BC$. If $AB^2 + AC^2 = 2AD^2$, show that $EF = DF$.

\item (11N2) In triangle $ABC$, let $X$ and $Y$ be the midpoints of $AB$ and $AC$, respectively. On segment $BC$, there is a point $D$, different from its midpoint, such that $\angle XDY = \angle BAC$. Prove that $AD$ is perpendicular to $BC$.

\end{enumerate}

\subsubsection*{Coordinate geometry}

\begin{enumerate}

\item (16QII3) Let $S$ be the set of all points $A$ on the circle $x^2 + (y-2)^2 = 1$ so that the tangent line at $A$ has a non-negative $y$-intercept; then $S$ is the union of one or more circular arcs. Find the total length of $S$.

\item (15AI7) Find the area of the triangle having vertices $A(10, -9), B(19,3),$ and $C(25,-21)$.

\item (16AII3) Point $P$ on side $BC$ of triangle $ABC$ satisfies $BP : PC = 2 : 1$. Prove that the line $AP$ bisects the median of triangle $ABC$ drawn from vertex $C$.

\end{enumerate}

\end{document}