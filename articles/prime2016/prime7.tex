\documentclass[10pt,paper=letter]{scrartcl}
\usepackage[alttitle]{cjquines}

\begin{document}

\title{VCSMS PRIME}
\subtitle{Session 7: Geometry 1}
\author{compiled by Carl Joshua Quines}
\date{October 12, 2016}

\maketitle

\subsubsection*{Circles}

\begin{enumerate}

\item (16QI5) The two internal tangents of two non-overlapping circles of radii $2$ and $4$ units intersect at right angles. What is the distance between the centers of the circles?

\item (16QI9) Triangles $PQR$ and $QRS$, where $P \neq S$, are two right triangles sharing the hypotenuse $QR$. If $C_1$ is the circle passing through $P, Q,$ and $R,$ and $C_2$ is the circle passing through $Q, R,$ and $S$, what can be said about $C_1 \cap C_2$?

\begin{enumerate}

\item[(a)] $C_1 \cap C_2$ consists of exactly one point.
\item[(b)] $C_1 \cap C_2$ consists of exactly two distinct points.
\item[(c)] $C_1 \cap C_2$ is empty.
\item[(d)] $C_1 \cap C_2$ consists of infinitely many points.

\end{enumerate}

\item (14QI15) Let $AB$ be a chord of circle $C$ with radius $13$. If the shortest distance of $AB$ to point $C$ is $5$, what is the perimeter of $ABC$?

\item (15AI18) Segment $CD$ is tangent to the circle with center $O,$ at $D$. Point $A$ is in the interior of the circle, and segment $AC$ intersects the circle at $B$. If $OA = 2, AB = 4, BC = 3$ and $CD = 6$, find the length of segment $OC$.

\item (14AII1) Two circles $C_1$ and $C_2$ of radius $12$ have their centers on each other. $A$ is the center of $C_1$ and $AB$ is a diameter of $C_2$. A smaller circle is constructed tangent to $AB$ and the two given circles, externally to $C_1$ and internally to $C_2$. Find the radius of this smaller circle.

\item (16ND4) Cyclic quadrilateral $ABCD$ has $DA = BC = 2$ and $AB = 4$. If $CD > AB$ and the lines $DA$ and $BC$ intersect at an angle of $60\dg$, find the radius of the circumscribing circle.

\end{enumerate}

\subsubsection*{Angles}

\begin{enumerate}

\item (16QI10) In triangle $ABC, BD$ is the angle bisector of $\angle ABC$ such that $AB = BD$. Point $E$ is on $AB$ such that $AE = AD$. If $\angle ACB = 36\dg$, find $\angle BDE$.

\item (16NE13) In parallelogram $ABCD, \angle BAD = 76\dg.$ Side $AD$ has midpoint $P,$ and $\angle PBA = 52\dg$. Find $\angle PCD$.

\item (14QII10) In $ABC,$ point $D$ is on $AC$ such that $AB = AD$ such that $\angle ABC - \angle ACB = 45\dg$. Find $\angle CBD$.

\item (15AI20) Trapezoid $ABCD$ has right angles at $C$ and $D$, and $AD > BC$. Let $E$ and $F$ be points on $AD$ and $AB$, respectively, such that $\angle BED$ and $\angle DFA$ are right angles. Let $G$ be the point of intersection of the segments $BE$ and $DF$. If $\angle CED = 58\dg$ and $\angle FDE = 41\dg$, what is $\angle GAB$?

\item (13ND2) Circles $A$ and $B$ are tangent to each other externally at $E$. The segment $CG$ is tangent externally to both circles, with $C$ on circle $A$ and $G$ on circle $B$. A point $D$ is selected on circle $A$ and $F$ on circle $B$ such that $D, E,$ and $F$ are collinear. The measure of minor arc $CE$ is $102\dg$. Find the measure of $\angle DFG$.

\end{enumerate}

\subsubsection*{Three-dimensional}

\begin{enumerate}

\item (13QI4) A convex polyhedron has $30$ faces and $62$ edges. How many vertices does this polyhedron have?

\item (14NE15) Six matchsticks, each $1$ unit long, are used to form a pyramid having equilateral triangles for its $4$ faces. What is the volume of this pyramid?

\item (14AI2) What is the largest number of $7 \times 9 \times 11$ boxes that can fit inside a box of size $17 \times 37 \times 27$?

\item (16NA9) $120$ unit cubes are put together to form a rectangular prism whose six faces are then painted. This leaves $24$ unit cubes without any paint. What is the surface area of the prism?

\item (11QIII4) Four spheres, each of radius $1.5$, are placed in a pile with three at the base and the other at the top. If each sphere touches the other three spheres, give the height of the pile.

\end{enumerate}

\subsubsection*{Areas}

\begin{enumerate}

\item (14NE3) A horse is tied outside a fenced triangular garden at one of the vertices. The triangular fence is equilateral with side length equal to $8$ units. If the rope with which the horse is tied is $10$ units long, find the area over which the horse can graze outside the fence assuming that the rope and the fence are strong enough to hold the animal.

\item (16NE10) Let $ABCD$ be a trapezoid with parallel sides $AB$ and $CD$ of lengths $6$ units and $8$ units, respectively. Let $E$ be the point of intersection of the extensions of the nonparallel sides of the trapezoid. If the area of $BEA$ is $60$ square units, what is the area of $BAD$?

\item (14NE6) In rectangle $ABCD$, point $E$ is chosen in the interior of $AD$, and point $F$ is chosen in the interior of $BC$. Let $AF$ and $BE$ meet at $G$, and $CE$ and $DF$ at $H$. The following areas are known: $[AGB] = 9, [BGF] = 16, [CHF] = 11, [CHD] = 15$. Find $[EGFH]$.

\item (13QIII1) $ABCD$ is a trapezoid with $AB||CD, AB = 6$ and $CD = 15$. If the area of $AED$ is $30$, find the area of $AEB$.

\item (16AI8) A side of an equilateral triangle is the diameter of a circle. If the radius of the circle is $1$, find the area of the region inside the triangle but outside the circle.

\item (16NA5) Square $ABCD$ with side length $2$ units has $M$ and $N$ as the midpoints of $AD$ and $BC$. Point $P$ is the intersection of segments $AN$ and $BM$, and lines $CP$ and $DP$ meet side $AB$ at points $Q$ and $R$. Find the sum of the areas of triangles $AMP, BNP, CPD$ and $PQR$.

\item (14AI15) Rectangle $BRIM$ has $BR = 16$ and $BM = 18$. The points $A$ and $H$ are located on $IM$ and $BM,$ respectively, so that $MA = 6$ and $MH = 8$. If $T$ is the intersection of $BA$ and $IH$, find the area of quadrilateral $MATH$.

\item (14AI17) Trapezoid $ABCD$ has parallel sides $AB$ and $CD$, with $BC$ perpendicular to them. Suppose $AB = 13, BC = 16$ and $DC = 11$. Let $E$ be the midpoint of $AD$ and $F$ the point on $BC$ so that $EF$ is perpendicular to $AD$. Find the area of quadrilateral $AEFB$.

\item (14AI20) The base $AB$ of a triangular piece of paper $ABC$ is $16$ units long. The paper is folded down over the base, with the crease $DE$ parallel to the base of the paper. The area of the triangle that projects below the base is $16\%$ the area of $ABC$. What is the length of $DE$?

\item (11AI18) A circle with center $C$ and radius $r$ intersects the square $EFGH$ at $H$ and at $M$, the midpoint of $EF$. If $C, E$ and $F$ are collinear and $E$ lies between $C$ and $F$, what is the area of the region outside the circle but inside the square in terms of $r$?

\item (13N2) Let $P$ be a point in the interior of $ABC$. Extend $AP, BP$ and $CP$ to meet $BC, AC$ and $AB$ at $D, E$ and $F$, respectively. If $APF, BPD$ and $CPE$, have equal areas, prove that $P$ is the centroid of $ABC$.

\end{enumerate}

\end{document}