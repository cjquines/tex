\documentclass[10pt,paper=letter]{scrartcl}
\usepackage[alttitle]{cjquines}

\begin{document}

\title{VCSMS PRIME}
\subtitle{Session 4: Combinatorics 1}
\author{compiled by Carl Joshua Quines}
\date{September 30, 2016}

\maketitle

\subsubsection*{Ad hoc}

\begin{enumerate}

\item (16QII9) Find the $2015$th digit in $122333444455555\ldots$

\item (15AI14) In how many ways can Alex, Billy and Charles split $7$ identical marbles among themselves so that no two have the same number of marbles? It is possible for someone not to get any marbles.

\item (15AI15) In a word finding game, a player tries to find a word in a $12 \times 12$ array of letters by looking at blocks of adjacent letters are arranged horizontally, arranged vertically or arranged diagonally. How many such $3$-letter blocks are there in a given $12\times 12$ array of letters?

\item (10NA2) The positive integers are grouped as follows: $A_1 = \{1\}, A_2 = \{2, 3, 4\}, A_3 = \{5, 6, 7, 8, 9\}$ and so on. In which group does $2009$ belong to?

\item (13NA1) How many pairs of diagonals on the surface of a rectangular prism are skew?

\item (13AII3) Let $v(X)$ be the sum of the elements of a set $X$. Calculate the sum of all numbers $v(X)$ where $X$ ranges over all non-empty subsets of the set $\{1,2,3,\ldots,16\}$.

\item (16ND1) The irrational number $0.123456789101112\ldots$ is formed by concatenating, in increasing order, all the positive integers. Find the sum of the first $2016$ digits of this number after the decimal point.

\item (16ND3) In an $n \times n$ checkerboard, the rows are numbered $1$ to $n$ from top to bottom, and the columns are numbered $1$ to $n$ from left to right. Chips are to be placed on this board so that each square has a number of chips equal to the absolute value of the difference of the row and column numbers. If the total number of chips placed on the board is $2660,$ find $n$.

\item (16AI20) Let $s_n$ be the sum of the digits of a natural number $n$. Find the smallest value of $\dfrac{n}{s_n}$ if $n$ is a four-digit number.

\item (9ND5) The vertices of a cube are each colored by either black or white. Two colorings of the cube are said to be \emph{geometrically the same} if one can be obtained from the other by rotating the cube. In how many geometrically different ways can the coloring be done?

\end{enumerate}

\subsubsection*{Inclusion-Exclusion}

\begin{enumerate}

\item (13AI15) There are $100$ people in a room. $60$ of them claim to be good at math, but only $50$ are actually good at math. If $30$ of them correctly deny that they are good at math, how many people are good at math but refuse to admit it?

\item (16QI6) How many positive integers less than or equal to $2015$ are divisible by $3$, but are neither divisible by $5$ nor $7$?

\item (9AI17) How many integers between $2$ and $10000$ do not share a prime factor with $10000$?

\item (16QII8) How many integers $x$ are there, where $100 \leq x \leq 2015$, and $x$ is divisible by $3$ or $8$, but not by $6$?

\item (10QII3) How many distinct natural numbers less than $1000$ are multiples of $10, 15, 35$ or $55$?

\end{enumerate}

\subsubsection*{Permutations}

\begin{enumerate}

\item (14NE13) How many three digit positive integers are there, the sum of whose digits is a perfect cube?

\item (16QI11) In how many ways can the letters of the word QUALIFYING be arranged such that the vowels are all in alphabetical order?

\item (15AI8) How many ways can $6$ boys and $6$ girls be seated in a circle so that no two boys sit next to each other?

\item (16AI15) In how many ways can the letters of the word ALGEBRA be arranged if the order of the vowels must remain unchanged?

\item (14AI13) How many positive integers, not having the digit $1$, can be formed if the product of all its digits is to be $33750$?

\item (14QI11) If all the words obtained from permuting the letters of the word $SMART$ are arranged alphabetically, what is the rank of the word SMART?

\item (14ND4) In how many ways can the letters in the word PHILLIP be arranged so that both of the strings PHI or ILL do not appear?

\item (11QII10) In how many ways can the letters of the word MURMUR be arranged without letting two letters which are the same be adjacent?

\end{enumerate}

\subsubsection*{Combinations}

\begin{enumerate}

\item (14NE5) A set $S$ has $n$ elements. There are exactly $57$ subsets of $S$ with two or more elements. How many elements does $S$ have?

\item (10QI6) How many ways can three distinct numbers be selected from the set $\{1, 2, 3, \ldots, 9\}$ if the product of these numbers is divisible by $21$?

\item (13AI12) Six boy-girl pairs are to be formed from a group of six boys and six girls. In how mnay ways can this be done?

\end{enumerate}

\subsubsection*{Balls and urns}

\begin{enumerate}

\item (16QI14) Mary wants to give $15$ cookies to Amy, Bob and Charlie. How many ways can she distribute the cookies to them such that Amy must receive at least $5$ cookies while Bob and Charlie must each receive at least $1$?

\item (16AI9) How many ways can you place $10$ identical balls in $3$ baskets of different colors if it is possible for a basket to be empty?

\item (16NE11) How many solutions does the equation $x+y+z=2016$ have, where $x, y$ and $z$ are integers with $x > 1000, y > 600$ and $z > 400$?

\item (10NE7) How many ways can you choose four integers from the set $\{1, 2, 3, \ldots, 10\}$ so that no two of them are consecutive?

\item (13QIII4) In how many ways can one select five books from a row of twelve books so that no two adjacent books are chosen?

\item (16QII5) The numbers $1, 2, \ldots, 12$ have been arranged in a circle. In how many ways can five numbers be chosen from this arrangement so that no two adjacent numbers are selected?

\item (16QIII4) Let $N = \{0, 1, 2, 3, \ldots\}$. Find the cardinality of the set $$\{(a,b,c,d,e)\in N^5 : 0 \leq a + b \leq 2, 0 \leq a + b + c + d + e \leq 4 \}.$$

\end{enumerate}

\end{document}