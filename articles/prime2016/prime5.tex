\documentclass[10pt,paper=letter]{scrartcl}
\usepackage[alttitle]{cjquines}

\begin{document}

\title{VCSMS PRIME}
\subtitle{Session 5: Algebra 2}
\author{compiled by Carl Joshua Quines}
\date{October 5, 2016}

\maketitle

\subsubsection*{Equations}

\begin{enumerate}

\item (13QI7) Sixty men working on a job have done $1/3$ of the work in $18$ days. The project is behind schedule and must be accomplished in the next twelve days. How many more workers need to be hired?

\item (14AI14) Solve the equation $(2-x^2)^{x^2-3\sqrt{2}x+4} = 1$.

\item (11NE4) If $\dfrac{x-a-b}{c} + \dfrac{x-b-c}{a} + \dfrac{x-c-a}{b} = 3$, where $a, b, c \in \RR^+$, find $x$ in terms of $a, b$ and $c$.

\item (14AI19) Find the values of $x$ in $(0, \pi)$ that satisfy the equation $$(\sqrt{2014} - \sqrt{2013})^{\tan^2 x} + (\sqrt{2014} + \sqrt{2013})^{-\tan^2 x} = 2(\sqrt{2014} - \sqrt{2013})^3$$.

\item (14QI7) For which $m$ does the equation $\dfrac{x-1}{x-2} = \dfrac{x-m}{x-6}$ have no solution in $x$?

\item (16AI3) Determine all values of $k \in \RR$ for which $\dfrac{4(2015^x) - 2015^{-x}}{2015^x - 3(2015^{-x})} = k$ admits a real solution.

\item (11ND4) Give three real roots of $\sqrt{x+3-4\sqrt{x-1}} + \sqrt{x+8 - 6\sqrt{x-1}} = 1$.

\item (13NE5) Find the solution set of the equation $(\sqrt{2} - 1)^x + 8(\sqrt{2} + 1)^x = 9$.

\end{enumerate}

\subsubsection*{Systems of equations}

\begin{enumerate}

\item (8NE1) If $wxy = 10, wyz = 5, wxz = 45, xyz = 12$, what is $w + y$?

\item (16AI12) Find all real solutions to the system of equations $x(y-1) + y(x+1) = 6, (x-1)(y+1) = 1$.

\item (13QI14) If $(a, b)$ is the solution of $\sqrt{x+y} + \sqrt{x-y} = 4, x^2-y^2=9,$ then find the value of $\dfrac{ab}{a+b}$.

\item (14NA5) Suppose that $w+4x+9y+16z = 6, 4w+9x+16y+25z=7, 9w+16x+25y+36z=12.$ Find $w+x+y+z$.

\item (10NA1) The nonzero numbers $x, y, z$ satisfy $xy = 2(x+y), yz = 4(y+z), xz = 8(x+z)$. Solve for $x$.

\end{enumerate}

\subsubsection*{Complex numbers}

\begin{enumerate}

\item (11AI8) Find all complex numbers $x$ satisfying $x^3 + x^2 + x + 1 = 0$.

\item (15AI3) Simplify the expression $\left(1 + \dfrac{1}{i} + \dfrac{1}{i^2} + \cdots + \dfrac{1}{i^{2014}}\right)^2.$

\item (13AI1) Find all complex numbers $z$ such that $\dfrac{z^4+1}{z^4-1} = \dfrac{i}{\sqrt{3}}.$

\item (13AI3) If $z^3 - 1 = 0$ and $z \neq 1$, find the value of $z + \dfrac{1}{z} + 4$.

\end{enumerate}

\subsubsection*{Polynomials}

\begin{enumerate}

\item (13NE7) Let $P(x) = ax^7 + bx^3 + cx - 5$, where $a, b,$ and $c$ are constants. If $P(-7) = 7$, what is $P(7)$?

\item (13NE1) In solving a problem that leads to a quadratic equation, one student made a mistake in the constant term only, obtaining the roots $8$ and $2$, while another student made a mistake in the coefficient of the first degree term, obtaining the roots $-9$ and $-1$. What was the original equation?

\item (13AI5) Consider a function $f(x) = ax^2 + bx + c, a > 0$, with two distinct roots a distance $p$ apart. By how much, in terms of $a, b, c$, should the function be translated downwards so that the distance between the roots becomes $2p$?

\item (15AI9) Two numbers $p$ and $q$ are both chosen randomly and independently of each other from the interval $[-2,2]$. Find the probability that $4x^2 + 4px + 1 - q^2 = 0$ has imaginary roots.

\item (9NA4) What is the coefficient of $x^5$ in the expansion of $(2 - x + x^2)^4$?

\item (13AII2) The quartic polynomial $P(x)$ satisfies $P(1) = 0$ and attains its maximum value of $3$ at both $x = 2$ and $x = 3$. Find $P(5)$.

\item (10NA8) When $(x^2 + 2x + 2)^{2009} + (x^2 - 3x - 3)^{2009}$ is expanded, what is the sum of the coefficients of the terms with odd exponents of $x$?

\item (16QIII5) How many terms are there when the expression of $(x+y+z)^{2015} + (x-y-z)^{2015}$ is expanded and simplified? 

\end{enumerate}

\subsubsection*{Polynomial factors}

\begin{enumerate}

\item (10NE11) Find the values of $a$ and $b$ such that $ax^4 + bx^2 + 1$ is divisible by $x^2 - x - 2$.

\item (13NE9) If $x^2 + 2x + 5$ is a factor of $x^4 + ax^2 + b$, find the sum $a + b$.

\item (13QII9) Factorize $(r-s)^3 + (s-t)^3 + (t-r)^3$.

\item (16QII6) How many (nonconstant) polynomial factors with leading coefficient $1$, with the other coefficients possibly complex, does $x^{2015} + 18$?

\item (11QIII2) Find all polynomials $p(x)$ where $xp(x-1)=(x-5)p(x)$ and $p(6) = 5!$.

\end{enumerate}

\subsubsection*{Remainder theorem}

\begin{enumerate}

\item (14AI11) Let $r$ be some real constant, and $P(x)$ a polynomial which has remainder $2$ when divided by $x-r,$ and remainder $-2x^2-3x+4$ when divided by $(2x^2+7x-4)(x-r)$. Find all values of $r$.

\item (11AI5) Let $f(x)$ be a cubic polynomial. If $f(x)$ is divided by $2x+3$, the remainder is $4$, while if it is divided by $3x+4$, the remainder is $5$. What will be the remainder when $f(x)$ is divided by $6x^2 + 17x + 22$?

\item (9NA7) Let $P(x)$ be a polynomial, that, when divided by $x - 19$, has the remainder $99$, and when divided by $x - 99$, has the remainder $19$. What is the remainder when $P(x)$ is divided by $(x - 19)(x- 99)$?

\end{enumerate}

\subsubsection*{Root-finding}

\begin{enumerate}

\item (16QII2) What is the difference between the largest and smallest real zeros of the function $f(x) = 2x^4 - 7x^3 + 2x^2 + 7x + 2$?

\item (13NA8) There are values of $m$ for which $x^2 - 2x(1+3m) + 7(3+2m) = 0$ has equal roots. What are these equal roots?

\item (13AI11) Let $f$ be a polynomial function that satisfies $f(x-5) = -3x^2 + 45x - 108$. Find the roots of $f(x)$.

\item (14AII3) If $p$ is a real constant such that the roots of the equation $x^3 - 6px^2 + 5px + 88 = 0$ form an arithmetic sequence, find $p$.

\item (11NE11) $x^3 + kx - 128 = 0$ has a root of multiplicity $2$. Find $k$.

\item (11AI17) Find all real numbers $a$ such that $x^3 + ax^2 - 3x - 2$ has exactly two distinct real zeros.

\end{enumerate}

\subsubsection*{Vieta's}

\begin{enumerate}

\item (16NE4) There are two distinct real numbers which are larger than their reciprocals by $2$. Find the product of these numbers. 

\item (16NA4) Let $f(x)$ be a polynomial function of degree $2016$ whose $2016$ zeros have a sum of $S$. Find the sum of the $2016$ zeros of $f(2x-3)$ in terms of $S$.

\item (9QII1) The roots of the quadratic equation $x^2 - 51x + k = 0$ differ by $75$, where $k$ is a real number. Determine the sums of the squares of the roots.

\item (11NE10) Find a quadratic polynomial with integer coefficients whose roots are the reciprocals of $x^2 + 4x + 8 = 0$.

\item (9NE13) Find the sum of the reciprocals of the roots of $4x^4 - 3x^3 - x^2 + 2x - 6 = 0$.

\item (13QII5) The equation $x^2 - bx + c$ has two roots $p$ and $q$. If the product $pq$ is to be maximum, what value of $b$ will make $b+c$ minimum?

\item (14NE11) Suppose $a, b, c$ are the roots of $x^3 - 4x + 1 = 0$. Find the value of $\dfrac{a^2bc}{a^3+1} + \dfrac{ab^2c}{b^3+1} + \dfrac{abc^2}{c^3+1}.$

\end{enumerate}

\subsubsection*{Coordinate plane}

\begin{enumerate}

\item (11QI4) For what values of $a$ does the system $x^2 - y^2 = 0, (x-a)^2 + y^2 = 0$ have a unique solution?

\item (10NE6) If the parabola $y + 1 = x^2$ is rotated clockwise by $90\dg$ about its focus, what will be the new coordinates of its vertex?

\item (14AI8) For what real values of $p$ will the graph of the parabola $y = x^2 - 2px + p + 1$ be on or above that of the line $y = -12x + 5$?

\item (14QII4) Let $(a, b)$ and $(c, d)$ be the two distinct points of intersection of circles $C_1$ and $C_2$. The circle $C_1$ is centered at the origin and passes through $P(16,16)$, while the circle $C_2$ is centered at $P$ and passes through the origin. Find $a+b+c+d$.

\item (13NA9) Let $A(-3, 0), B(3, 0), C(0, 5)$ and $D(0, -5)$. How many points $P(x, y)$ on the plane satisfy both $PA + PB = 10$ and $|PC - PD| = 6$?

\item (11QII7) A line with $y$-intercept $5$ and positive slope is drawn such that the line intersects $x^2+y^2=9$. What is the least slope of such a line?

\item (16AI17) Find the area of the region bounded by the graph of $\abs{x} + \abs{y} = \dfrac{1}{4}\abs{x+15}$.

\item (13AI4) Find the equation of the line that contains the point $(1, 0)$, that is of least positive slope, and that does not intersect the curve $4x^2 - y^2 - 8x = 12$.

\item (13AI6) Find the equation of the circle, in the form $(x - h)^2 + (y - k)^2 = r^2$, inscribed in a triangle whose vertices are located at the points $(-2, 1), (2, 5), (5, 2)$.

\item (13NE12) What is the length of the shortest path that begins at the point $(-3, 7)$, touches the $x$-axis, and then ends at a point on the circle $(x-5)^2 + (y-8)^2 = 25$?

\end{enumerate}

\end{document}