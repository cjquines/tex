\documentclass[10pt,paper=letter]{scrartcl}
\usepackage[alttitle]{cjquines}

\begin{document}

\title{VCSMS PRIME}
\subtitle{Session 7: Geometry 1}
\author{compiled by Carl Joshua Quines}
\date{October 12, 2016}

\maketitle

\subsubsection*{Circles}

\begin{enumerate}

\item Let the centers of the circles be $A, B$, one internal tangent be $CD$ tangent to circle $A$ at $C$, and to circle $B$ at $D$, and let $E$ be the intersection of the two tangents. 

Since $\angle E$ is right, and $\angle C$ is right as well, then $ACE$ must be an isosceles right triangle. Thus $AE, BE$ are $4\sqrt2$ and $2\sqrt2$, so $AB$ is $6\sqrt2$.

\item Since $\angle QPR + \angle QSR = 180\dg$ then quadrilateral $PQRS$ is cyclic, so $C_1$ and $C_2$ are the same circle, and they intersect at infinitely many points. 

\item Drop the perpendicular from $C$ to $AB$ at point $D$. Then $CD = 5$, and $CB = 13$, so by Pythagorean, $DB = 12$. Similarly, $AD = 12$, so the perimeter is $12 + 12 +13 + 13 = 50$.

\item Extend $CB$ to meet the circle again at $F$. By power of a point, we get $CF = 12$, and so $AF = 5$. By Stewart's on triangle $OBF$ we find $OB = BF = 2\sqrt6$. Pythagorean on $OCD$ gives $OC = 2\sqrt{15}$.

\item Let $C_2$ have center $O$, the smaller circle have center $P$, tangent to $C_1, C_2$ and $AB$ at $I, J, K$, respectively. Let the smaller circle have radius $r$ and let $OK = s$.

Then Pythagorean on $APK$ gives $AP^2 = AK^2 + PK^2$, or $(AI + IP)^2 = (AO + OK)^2 + PK^2$, or $(12 + r)^2 = (12 + s)^2 + r^2$. Pythagorean on $OPK$ gives $OP^2 = PK^2 + OK^2$, or $(OJ - JP)^2 = PK^2 + OK^2$, or $(12 - r)^2 = r^2 + s^2$. The $r^2$ term cancels in both equations, and we can equate $24r$ in both to get $144 - s^2 = s^2 + 24s$. Thus $s = 6\sqrt3 - 6$ and $r = 3\sqrt3$.

\item From power of a point on $E$ we get $AE = BE$ so $ABE$ is equilateral. Thus $\angle ABC = 120\dg$. By the law of cosines $AC = 2\sqrt7$, and by the extended law of sines $2R = \dfrac{AC}{\sin120\dg}$, so the circumradius is $\dfrac{2\sqrt{21}}3$.

\end{enumerate}

\subsubsection*{Angles}

\begin{enumerate}

\item Let $\angle ABD = \angle DBC = x\dg$. We know that $\angle ADB = \angle DBC + \angle BCD$ since it is an exterior angle, however $\angle ADB = \dfrac{180\dg - \angle ABD}2$ as triangle $ABD$ is isosceles. Equating gives $x + 36 = \dfrac{180 - x}2$, or $x = 36\dg$. Thus since triangle $ABD$ is isosceles, $\angle ADB = \dfrac{180\dg - \angle ABD}2 = 72\dg$; since triangle $ADE$ is isosceles $\angle ADE = \dfrac{180\dg - \angle DAB}2 = 54\dg$, and so $\angle BDE = \angle ADB - \angle ADE = 17\dg$.

\item Let $Q$ be the midpoint of $BC$. Then $\angle ABP = \angle APB = 52\dg$ by triangle angle sum on $ABP$, so $AB = BP$. Then $ABQP$ is a rhombus. Then $AQ$ is an angle bisector since it is a diagonal, so $\angle AQP = 38\dg$. But $PC || AQ$ and $PQ || CD$ so $\angle PCD = \angle AQP = 38\dg$.

\item Note $\angle CBD = \angle ADB - \angle DCB$ upon considering exterior $\angle ADB$. But $\angle ADB = \angle ABD = \angle ABC - \angle CBD$ through isosceles triangle $ABD$. Substituting, $\angle CBD = \del{\angle ABC - \angle CBD} - \angle DCB = \del{\angle ABC - \angle ACB} - \angle CBD = 45\dg - \angle CBD$. Thus $\angle CBD = 22.5\dg$.

\item Since in $AFGE$ we have $\angle AFG + \angle AEG = 90\dg + 90\dg = 180\dg$, it is a cyclic quadrilateral. Similarly, since in $BDEF$ we have $\angle BED = \angle BFD = 90\dg$ then it is also cyclic. Thus $\angle GAB = \angle GAF$, and $\angle GAF = \angle GEF$ by cyclic quadrilateral $AFGE$, and $\angle GEF = \angle BEF = \angle BDF$ by cyclic quadrilateral $BDEF$. However, $\angle BDF + \angle FDE = \angle CED$ since $BCDE$ is a rectangle. Thus $\angle GAB = \angle BDF = 17\dg$.

\item From $CA \perp CG$ and $BG \perp CG$ we have $CA || BG$. Then $\angle ABG + \angle CAB = 180\dg$, whence $\angle ABG = 78\dg$. Then $\angle ABG = \angle EBG = 2\angle EFG = 2\angle DFG$, so $\angle DFG = 39\dg$.

\end{enumerate}

\subsubsection*{Three-dimensional}

\begin{enumerate}

\item By Euler's formula, $V - E + F = 2$, so $V = 34$.

\item It is a regular tetrahedron of edge $1$. Drop the height from the top vertex to the base, which hits its center. It forms a right triangle with one edge as the hypotenuse, the other leg is from the length from a vertex to the center. The other leg is $2/3$ the median, so its length is $\dfrac{\sqrt3}3$. This gives its height as $\sqrt{1^2 - \del{\dfrac{\sqrt3}3}^2} = \dfrac{\sqrt6}3$. Its volume is one-third the area of the base times its height, or $\dfrac13\cdot\dfrac{\sqrt3}4\cdot\dfrac{\sqrt6}3 = \dfrac{\sqrt2}{12}$.

\item We stack the $7 \times 9 \times 11$ boxes in a $2 \times 3 \times 3$ fashion, making it take up $14 \times 27 \times 33$, which fits in the $17 \times 27 \times 37$ box. This makes the maximum number $18$.

\item Let the sides of the prism be $x, y, z$; we have $xyz = 120$ and $(x-2)(y-2)(z-2) = 24$. WLOG $z$ is divisible by $5$. Then if $z = 5$, we see $(6, 4, 5)$ works. The surface area is then $2(6\cdot5 + 5\cdot4 + 4\cdot6) = 148$.

\item The centers of the spheres form a regular tetrahedron of edge $3$. Through similar logic as number 2 in this section, its height is $\sqrt{3^2 - \sqrt{3}^2} = \sqrt6$. The overall height is the height of the tetrahedron plus two radii, so its height is $3 + \sqrt6$.

\end{enumerate}

\subsubsection*{Areas}

\begin{enumerate}

\item The area consists of two $150\dg$ sectors of a circle with radius $10$, one on either side of the horse. Wrapping around the equilateral triangle gives two more $120\dg$ sectors, of radius $10 - 8 = 2$. The total area is thus $2 \cdot \dfrac{150\dg}{360\dg} \pi \cdot 10^2 + 2 \cdot \dfrac{120\dg}{360\dg} \pi \cdot 2^2 = 86\pi$.

\item Drop the altitude from $E$ to $AB$ and $CD$, which are parallel, so the altitude is the same line. The length of the altitude from $E$ to $AB$ has to be $20$ for the area of $AEB$ to be $60$. Since $AB || CD$ we have $EAB \ sim EDC$ and thus the length of the altitude from $E$ to $CD$ has to be $\dfrac{80}3$. Thus the distance between lines $AB$ and $CD$ is $\dfrac{80}3 - 20 = \dfrac{20}3$, which is also the length of the altitude from $D$ to $AB$. Thus $[BAD] = \dfrac12 \cdot 6 \cdot \dfrac{20}3 = 20$.

\item Note that $\triangle AEB$ and $\triangle AEF$ share the same base and altitude, so they have the same area. Subtracting $[AEG]$ from both gives $[ABG] = [EFG] = 9$. Similarly, $[CDH] = [EFH] = 15$. Thus $[EGFH] = [EFG] + [EFH] = 24$.

\item (Should have $E$ as intersection of diagonals.) Note that $AEB$ and $CED$ are similar with ratio $6 : 15$. Then $EB : ED = 6 : 15$ as well, as $AED$ and $AEB$ share the same altitude from $A$, their areas are in the ratios of their bases, so $[AEB] : [AED] = 6 : 15$. Thus $[AEB] = 12$.

\item Let the triangle be $ABC$ intersecting the circle with center $O$ at $B'$ and $C'$ lying on $AB$ and $AC$, respectively. The required region is quadrilateral $AB'OC'$ minus the sector with arc $B'C'$. This is twice the area of a unit equilateral triangle minus the unit sector of $60\dg$, or $2 \cdot \dfrac{\sqrt{3}}4 - \dfrac16\pi = \dfrac{3\sqrt3 - \pi}6$.

\item In rectangle $ABMN$ with area $2$, triangles $APM$ and $BPN$ form half the area, so the sum of their areas is $1$. $P$ is vertically halfway between $AM$ and $BN$, so its distance to $DC$ is $\dfrac32$. The area of $DPC$ is thus $\dfrac12 \cdot 2 \cdot \dfrac32 = \dfrac32$. Then triangles $PQR$ and $DCP$ are similar, but the height from $P$ to $QR$ is the distance from $P$ to $AB$, which is $\dfrac12$. Thus the ratio of similarity is $1 : 3$, so the ratio of their areas is $1 : 9$, thus the area of $PQR$ is $\dfrac16$. The sum is $\dfrac83$.

\item It is simplest to Cartesian bash. Set $M(0, 0)$, $B(0, 18)$, $I(16, 0)$. Thus $H(0, 8)$ and $A(6, 0)$. Line $BA$ is $\dfrac{x}6 + \dfrac{y}{18} = 1$ in intercept form, also line $IH$ is $\dfrac{x}{16} + \dfrac{y}8 = 1$. Equating gives $\dfrac{x}6 - \dfrac{x}{16} = \dfrac{y}8 - \dfrac{y}{18}$ or $\dfrac{10x}{6 \cdot 16} = \dfrac{10y}{8 \cdot 18}$, cancelling gives $3x = 2y$. Substituting back to either equation gives $T(4, 6)$. Using the shoelace formula on $MATH$ gives its area as $34$.

\item It is also simple to Cartesian bash: set $C(0, 0)$, $B(0, 16)$, $A(13, 16)$ and $D(11, 0)$. Then $E$ is a midpoint so $E(12, 8)$. The slope of $AD$ is $8$ so the slope of $EF$ is $-\dfrac18$. Point $F$ lies on $BC$ so its $x$-coordinate is zero; it lies on $EF$ so $F(0, 9.5)$. Using the shoelace formula gives $91$.

\item Suppose that point $C$ is $C'$ after folding, and $DC'$ and $EC'$ intersect $AB$ at $A'$ and $B'$ respectively. Drop altitudes $H$ from $C$ to $DE$ and $M$ from $C$ to $AB$. Clearly $C, H, M, C'$ are collinear. The ratio $[A'B'C'] : [ABC] = 16 : 100$ is given, thus the ratio $C'M : CM = 4 : 10$ due to similarity. Also, $CH = C'H$ since they are the same altitude after folding. Since $CH + C'H = CM + C'M$ due to collinearity, $2CH = CM + \dfrac25CM$ from earlier. By similarity, $CH : CM = 7 : 10 = DE : AB$, so $DE = \dfrac{56}5$.

\item Let $x$ be the side of the square. The Pythagorean theorem on right $CEH$ gives $\del{r - \dfrac{x}2}^2 + x^2 = r^2$, so $x = \dfrac45 r$. Thus $\angle HCE = \tan^{-1}\dfrac43$. The required area is equal to $[CHGF]$ minus the sector with arc $HM$; the former is $\dfrac12\del{r + \dfrac{x}2 + x} x$ while the latter is $\dfrac12 r^2 \tan^{-1}\dfrac43$. Simplifying yields $r^2 \del{\dfrac{22}{25} - \dfrac12 \tan^{-1}\dfrac43}$.

\item Official solution uses algebra and whatever. We use Cartesian. Take an affine transformation to $A(0, 1)$, $B(1, 0)$, $C(0, 0)$ which preserves the problem, and let $P(a, b)$. It is easy to bash $D\del{-\dfrac{a}{b-1}, 0}$, $E\del{0, -\dfrac{b}{a-1}}$, $F\del{\dfrac{a}{a+b}, \dfrac{b}{a+b}}$. Then $[DBP] = [ECP] = [FAP]$ and bashing gives $a = b = \dfrac13$, which is as required.

\end{enumerate}

\end{document}