\documentclass[10pt,paper=letter]{scrartcl}
\usepackage[wide]{cjquines}

\begin{document}

\title{VCSMS PRIME}
\author{Valenzuela City School of Mathematics and Science}
\date{Program for Improving Mathematical Excellence}

\maketitle

\subsubsection*{Overview}

The Valenzuela City School of Mathematics and Science -- Program for Improving Mathematical Excellence (PRIME) is a year-round mathematics enrichment program. It is typically held after class and caters to high school students within the school. Its primary aims are:

\begin{itemize}

\item to introduce students to mathematics outside the curriculum,

\item to improve problem-solving skills, with a focus on competition mathematics, and

\item to provide a setting for communicating with a vibrant peer group.

\end{itemize}

PRIME began as the Enhanced Training Program for Mathematical Excellence (ETPME) in 2014. The ETPME was renamed as the Enhanced Training Program in Mathematics (ETPM) in 2015, and was finally renamed PRIME in 2016.

\subsubsection*{Tracks}

There are several tracks of PRIME that occur in a school year:

\begin{itemize}

\item June -- August: The first track of PRIME is for grade 7 students, aimed to screen potential contestants for future competitions. Roughly one to two dozen students participate. The material covers mostly introductory algebra: the laws of exponents, solving linear equations, and word problems.

\item August -- September: The second track of PRIME is aimed at preparing for the Australian Mathematics Competition. About one or two grade 7 students are picked from the first track to participate, while returning contestants from previous years participate in the training as well. Sessions mostly consist of practice tests and discussion of solutions.

\item September -- October: The third track of PRIME is aimed at preparing for the Qualifying Stage of the Philippine Mathematical Olympiad. About six returning contestants participate. Sessions consist of answering past questions, covering things not covered early in the curriculum such as logarithms, trigonometry, complex numbers, and others.

\item October -- February: The fourth track of PRIME is aimed at the short-answer oral contests held in this period of the year, including but not limited to AdMU's Sipnayan, UST's Pythagoras and UP's Mathirang Mathibay. Collectively, about a dozen different students participate. Sessions consist of answering timed questions in preparation for the oral rounds of these, mostly taken from past competitions.

\item January -- March: The final track of PRIME is aimed at the Metrobank-MTAP-DepEd Mathematics Competition. This is typically the most intensive track, as VCSMS has a track record to uphold in the competition. Roughly a dozen students per grade level are put in the initial pool. Sessions are aimed at preparing for the written and oral rounds of the competition, and thus consist of answering past written or oral questions.

\end{itemize}

\end{document}