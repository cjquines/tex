\documentclass[11pt,paper=letter]{scrartcl}
\usepackage[alttitle]{cjquines}

\begin{document}

\title{Guessing games}
\author{CJ Quines}
\date{April 27, 2024}

\maketitle

\subsubsection*{Warmup}

\begin{enumerate}
\item (Classical) Dylan and Sarji and playing a game. Let $A = \{0, 1, \ldots, 2^{10} - 1\}$. Dylan chooses an integer $a$ from $A$, and Sarji is trying to guess it. Sarji can specify a subset $S \subseteq A$, and Dylan replies by saying whether $a \in S$. What's the least number of questions Sarji needs to find $a$?
\begin{enumerate}[(a)]
\item Write $a$ in binary. How can Sarji learn one of $a$'s digits with each question?
\item Suppose Sarji only asks $9$ questions. How many sequences of $9$ answers can Sarji get? Show that Sarji has at least two possibilities for Dylan's number.
\end{enumerate}

\item (\href{https://artofproblemsolving.com/community/c6h2815100p24855547}{Iran TST 2022/1}) Morteza and Mahdi are playing a game. Morteza chooses $100$ finite sets of integers, $A_1, \ldots, A_{100}$, and Mahdi is trying to guess the elements of each set. Mahdi can say two integers $1 \le i < j \le 100$, and Morteza replies by giving the elements in $A_i \cap A_j$ and $A_i \cup A_j$. What's the least number of questions Mahdi needs to find the elements in each set?

\begin{enumerate}[(a)]
\item Show that if Mahdi asks for $(i, j)$, $(j, k)$, and $(k, i)$, Mahdi can find $A_i$, $A_j$, and $A_k$. Induct to show Mahdi can win with $100$ questions.
\item Suppose Mahdi only asks $99$ questions. Construct the graph $G$ with one vertex for each set, and draw an edge between two sets if Mahdi asks about them together. Show that some component of $G$ is a tree $T$.
\item Suppose Morteza answers, for each question corresponding to an edge in $T$, that the intersection is $\varnothing$ and the union is $\{0\}$. Show that Mahdi has at least two possibilities for the elements of each set corresponding to a vertex in $T$.
\end{enumerate}

\end{enumerate}

\subsubsection*{Thoughts}

Guessing games involve a \textbf{questioner}, who asks \textbf{queries} to an \textbf{answerer}, who has a \textbf{secret}. There's mostly two kinds of problems. (Unless it's not a combinatorics problem. Then what it's about probably won't be a secret.) Anyway.

\begin{itemize}
\item There's problems secretly about binary:
\begin{itemize}
\item We show the questioner can win with $q$ queries by considering the binary digits of the secret, or by learning half of the secret with each query.
\item We show the questioner can't win with $q - 1$ queries, often through a counting argument. If there's more possibilities for the secret than there are possibilities for answer sequences, by pigeonhole, there's some sequence of answers that correspond to at least two secrets.
\end{itemize}

\item And there's problems secretly about graphs (ish):
\begin{itemize}
\item We show the questioner can win with $q$ queries, often inductively. If the questioner wants to learn $n$ parts of the secret, show the questioner can learn $n - 1$ parts of it.
\item We show the questioner can't win with $q - 1$ queries by converting the problem to a graph. Usually we set it up so that each part of the secret is a vertex, and each query adds an edge. Then we show there's two possible secrets that correspond to the graph.
\end{itemize}
\end{itemize}

\subsubsection*{Examples}

\begin{enumerate}

\item (Classical) We have $n$ lamps and $n$ switches. Our goal is to decide which switch opens which lamp. In one move, we can specify a set of switches, toggle them, and observe which lamps are toggled. Find the least number of moves needed to learn which switch goes to each lamp.

\textbf{Sketch:} It's $\ceil{\log_2 n}$ moves. For simplicity, assume $n = 2^k$.

We show $k$ moves are enough. Number the switches with $k$-digit binary numbers. For the $i$th move, toggle all switches whose $i$th digit is $1$.

We show $k - 1$ moves aren't enough. The information we have for each lamp is which moves it was toggled in; there's only $2^{k-1}$ such sets of moves, but $2^k$ lamps.

\item (\href{https://artofproblemsolving.com/community/c6h5504p17843}{Russia 2004/9/3}) There are $2004$ boxes, each containing a white or a black ball. It is known that a non-zero even number of balls are white. A move consists of picking two boxes and asking whether at least one of them contains a white ball. What minimal number of questions is required, after which one can indicate two boxes both containing a white ball?

\textbf{Sketch:} It's $4005$.

We show $4005$ moves are enough. Ask $(1, 2)$, $(1, 3)$, etc., if any of them is a no, then $1$ is black and we're done. (Why?) Else $1$ is white. Repeat for $(2, 3)$, $(2, 4)$, etc. If $2$ is black we're done, else answer $1$ and $2$.

We show $4004$ moves aren't enough. Consider the graph with vertices as balls and edges as queries. Suppose we answer $u, v$. If $uv$ isn't an edge, they could both be black, contradiction. Find a vertex $w$ such that at most one of $uw$ and $vw$ is an edge; it exists because $4004 < 1 + 2002 + 2002$. If $uw$ edge, declare $v$ and $w$ black, $u$ white, contradiction.

\item (\href{https://artofproblemsolving.com/community/c6h514285p2889223}{Russia 2000/9/2}) Tanya chose a natural number $X\le100$, and Sasha is trying to guess this number. He can select two natural numbers $M$ and $N$ less than $100$ and ask Tanya about $\gcd(X+M,N)$. Show that Sasha can determine Tanya's number with at most seven questions.

\textbf{Sketch:} Write $X$ in binary, we'll find its digits right-to-left. Ask $\gcd(X + 2, 2)$. If the last digit is $1$, ask $\gcd(X + 1, 4)$, otherwise ask $\gcd(X + 4, 4)$. In general, if the last $k$ digits interpreted as binary is $k'$, ask $\gcd(X + 2^k - k', 2^{k + 1})$.

\textbf{Remark:} There's lots of other solutions, but I think this is most motivated.

\item (\href{https://artofproblemsolving.com/community/c6h596946p3542240}{K\"ursch\'ak 2010/1}) We have $n$ keys and $n$ locks. Our goal is to decide which key opens which lock. In one move, we choose a key and a lock, and test whether the lock can be opened with the key. Find the least number of moves needed to learn which key goes to each lock.

\textbf{Sketch:} It's $\binom{n}{2}$ moves.

We show $\binom{n}{2}$ moves are enough by induction. Try the first key on every lock; this takes at most $n$ moves. Then there are $n-1$ keys left to match with $n-1$ locks, by hypothesis this takes at most $\binom{n-1}{2}$ moves.

We show $\binom{n}{2} - 1$ moves aren't enough. Make a $K_{n, n}$ where one side is keys and the other side is locks; the secret corresponds to a (perfect) matching. Fail each query and remove its edge from the graph, unless we'd make a matching impossible.

After all queries we have a subgraph of $K_{n, n}$ with at least $n^2 - \left( \binom{n}{2} - 1 \right)$ edges. By our answers, we know one's matching possible. Show that a cycle exists for size reasons. Its symmetric difference with the matching is another matching.

\end{enumerate}

\subsubsection*{Problems}

These are mostly direct applications of the above ideas.

\begin{enumerate}

\item (\href{https://artofproblemsolving.com/community/c6h1829626p12247062}{HMIC 2019/2}) Annie has a permutation $(a_1, a_2, \dots ,a_{2019})$ of $S=\{1,2,\dots,2019\}$, and Yannick wants to guess her permutation. With each guess Yannick gives Annie an $n$-tuple $(y_1, y_2, \dots, y_{2019})$ of integers in $S$, and then Annie gives the number of indices $i\in S$ such that $a_i=y_i$. Show that Yannick can always guess Annie's permutation with at most $24000$ guesses.
\hint{\ref{h:1}}

\item (\href{https://artofproblemsolving.com/community/c6h1352101p7388263}{TOT Spring 2016/Junior A/7}) Let $n > 2$ be an integer. There are $n + 1$ good batteries and $n$ batteries. A lamp uses two batteries, and it works only if both of them are good.
\begin{enumerate}[(a)]
\item What is the least number of attempts needed to make the lamp work?
\item Same as (a), but with $n$ good batteries.
\end{enumerate}
\hint{\ref{h:2}}

\item (\href{https://artofproblemsolving.com/community/c6h2478238p20786985}{India 2021/4+}) A Magician and a Detective play a game. The Magician lays down cards numbered from $1$ to $52$ face-down on a table. On each move, the Detective can point to two cards and inquire if the numbers on them are consecutive. The Magician replies truthfully. After a finite number of moves, the Detective points to two cards. She wins if the numbers on these two cards are consecutive, and loses otherwise. What's the least number of questions the Detective needs to win?
\hint{\ref{h:3}}

\item (\href{https://artofproblemsolving.com/community/c6h1831665p12265499}{Russia 2019/11/3}) We are given $n$ coins of different weights and $n$ balances, for some $n > 2$. On each turn one can choose one balance, put one coin on the right pan and one on the left pan, and observe the result. However, one of the $n$ balances is broken, and only shows random results. What is the least number of turns needed to find the heaviest coin?
\hints{\ref{h:4} \ref{h:5}}

\item (\href{https://artofproblemsolving.com/community/c6h209617p1154356}{Russia 2008/11/6+}) A magician should determine the area of a hidden convex $2008$-gon $A_{1}A_{2} \cdots A_{2008}$. In each step he chooses two points on the perimeter, where the chosen points can be vertices or points dividing selected sides in selected ratios. Then his helper divides the polygon into two parts by the line through these two points and announces the area of the smaller of the two parts. Show that the magician can find the area of the polygon in $50$ steps.
\hints{\ref{h:6} \ref{h:7}}

\end{enumerate}

\subsubsection*{Other problems}

These are less direct applications. Not necessarily harder than the last section.

\begin{enumerate}[resume]

\item (\href{https://artofproblemsolving.com/community/c6h2494130p21012476}{Brazil 2020/5}) Let $n \ge k$ be positive integers. Each person in a group of $n$ people either always tells the truth or always lies. Arnaldo can ask one of these people ``In set $A$, are there an even number of people who always tell the truth?'' where $A$ is a subset of size $k$ of the set of $n$ people. The answer is always ``yes'' or ``no''.
\begin{enumerate}[(a)]
\item For which $n$ and $k$ is it possible to determine which people always tell the truth, and which people always lie?
\item If it's possible, what's the minimum number of questions needed?
\end{enumerate}
\hint{\ref{h:8}}

\item (\href{https://artofproblemsolving.com/community/c6h1191266p5811670}{UK 2016/2/2}) Alison has compiled a list of $20$ hockey teams, ordered by how good she thinks they are, but refuses to share it. Benjamin may mention three teams to her, and she will then choose either to tell him which she thinks is the weakest of the three, or which she thinks is the strongest of the three. Benjamin may do this as many times as he likes. Determine the largest $N$ such that Benjamin can guarantee to be able to find a sequence $T_{1}, T_{2}, \ldots, T_{N}$ of teams with the property that he knows that Alison thinks that $T_{i}$ is better than $T_{i+1}$ for each $1 \leq i < N$.
\hints{\ref{h:9} \ref{h:10}}

\item (\href{https://artofproblemsolving.com/community/c6h1312977p7040107}{TOT Spring 2016/Senior A/4} / \href{https://oj.uz/problem/view/IOI14_game}{IOI 2014/game}) There are $64$ towns in a country and some pairs of towns are connected by roads but we do not know these pairs. We may choose any pair of towns and find out whether they are connected or not. Our aim is to determine whether it is possible to travel from any town to any other by a sequence of roads. Prove that there is no algorithm which enables us to do so in less than $2016$ questions.
\hint{\ref{h:11}}

\item (\href{https://steamcommunity.com/sharedfiles/filedetails/?id=2509179914}{Wire association}) Let $n > 2$ be an integer, and $S = \{1, \ldots, n\}$. There's a secret permutation $\pi \from S \to S$. In a move, you choose a set $P$, whose elements are disjoint subsets of $S$. You then choose $p = \pi$ or $\pi^{-1}$, and learn $\{\{p(a) \mid a \in A\} \mid A \in P\}$. (You learn a set, so the elements are unordered.) What's the smallest number of moves needed to learn $\pi$?
\hints{\ref{h:12} \ref{h:13}}

\textbf{Remark:} Once you think you've found the answer, check your guess with this interactive version: \url{https://files.timwi.de/Misc/Wire%20puzzle.html}. You choose the partition by clicking some numbers and then ``make group.''

\item (\href{https://oj.uz/problem/view/IOI16_messy}{IOI 2016/messy}) Let $n > 2$ be an integer, and $S = \{1, \ldots, n\}$. There's a secret permutation $\pi \from S \to S$. First, you choose a set $P$, with $|P| \le \ceil{n\log_2 n}$, whose elements are subsets of $S$. Then you make several queries. In each query, you choose a subset $A$ of $S$, and learn if there exists some $A' \in P$ such that $A = \{\pi(a) \mid a \in A'\}$. Prove that you can learn $\pi$ with at most $\ceil{n\log_2 n}$ queries.
\hints{\ref{h:14} \ref{h:15}}

\item (\href{https://artofproblemsolving.com/community/c6h544046p3143005}{Russia 2013/11/4}) There are $2013$ cards face-down on a table, each with a distinct number. In one move, we can point to ten cards and learn the number on one of the cards, though we don't know which. What's the largest number of cards we can guarantee to identify the numbers of?
\hints{\ref{h:16} \ref{h:17} \ref{h:18}}

\end{enumerate}

\newpage

\subsubsection*{Hints}

\begin{enumerate}

\item \label{h:14} Pretend you could restart from the beginning. Split the problem in half.

\item \label{h:7} To find $P$, do some sort of ``ternary search''.

\item \label{h:16} There's only one way we can be sure of a card's number, which is if all queries that have that card as an answer intersect only in that card.

\item \label{h:9} Draw the graph with an edge between two vertices if we don't know who's stronger.

\item \label{h:3} For the lower bound, you can use Ore's theorem.

\item \label{h:18} If there are only $27$ cards, show there's a counterstrategy that prevents us from learning any cards, by grouping the cards into triangles.

\item \label{h:4} Pretend one balance is correct and use it to find the heaviest coin. Do the same thing with another balance.

\item \label{h:12} The answer is two moves.

\item \label{h:1} Find where $1$ is, then lamps-and-switches.

\item \label{h:15} How do you make sure queries from the smaller problems don't overlap? Add extra elements.

\item \label{h:5} For the lower bound, show that there's some coin that only got compared with once; what can we say about that coin?

\item \label{h:11} Answer no unless you have to. If we knew that the graph was connected before the last edge, then those two vertices must already be joined by some path.

\item \label{h:2} For the lower bound, you can use Tur\'an's theorem.

\item \label{h:13} If you know $\pi(a) = b$ and $\{\pi^{-1}(b), \pi^{-1}(c)\} = \{\pi(a), \pi(d)\}$\dots

\item \label{h:8} Ask everyone the same set.

\item \label{h:10} For the upper bound, answer so that pairs are indistinguishable.

\item \label{h:17} If there are only $18$ cards, show there's a counterstrategy that prevents us from learning any cards, by pairing the cards up.

\item \label{h:6} Find a point $P$ such that $A_1P$ divides the polygon in half.

\end{enumerate}

\newpage

\subsubsection*{Sketches}

\begin{enumerate}
\item Use $2018$ queries to find $a_1$, WLOG $a_1 = 2019$; for each subsequent query we'll fix $a_1 = 2019$. Use the lamps-and-switches method to get $1025$ through $2018$, each takes $11$ queries. Then use lamps-and-switches to get $1$ through $1024$, each takes at most $10$ queries.

\item For (a), answer is $n + 2$. Divide into pairs and query each. If all fail, then the remaining battery must be good, query it with both batteries of a pair. To show $n + 1$ isn't possible, need to show that a graph with $2n + 1$ vertices and $n + 1$ edges has an independent set of $n + 1$. Equivalently, its complement needs to have a clique of size $n + 1$. This follows from \href{https://en.wikipedia.org/wiki/Turán's_theorem}{Tur\'an's theorem}. (b) is similar, but you have to do a bit more work in the construction.

\item It's $50$, ask $1$ with $2$ through $51$, if there's a yes we're done, otherwise answer $1$ and $52$. If $\le 49$ queries, consider graph, want to show a Hamiltonian path exists in its complement. This follows from \href{https://en.wikipedia.org/wiki/Ore's_theorem}{Ore's theorem}. (If you haven't seen it before, the statement is provable by induction; a graph such that $\deg u + \deg v \ge |V|$ for all non-adjacent $u, v$ is Hamiltonian.)

\item It's $2n - 1$. Pick three balances $A, B, C$. Use $A$ to get the heaviest coin with $n - 1$ queries, similarly use $B$. If they agree we're done, otherwise use $C$ to compare. To show $2n - 2$ isn't enough, draw the graph with $u \to v$ if $u$ weighs less than $v$. Some vertex $v$ has outdegree $\le 1$. We can't distinguish between the world where all balances were correct, and the world where $v$ is the heaviest.

\item There's exactly one point $P$ on the perimeter such that $A_1P$ divides the polygon in half; say it's on side $A_iA_{i+1}$. Then, if $M$ is the midpoint of $A_iA_{i+1}$, we can use $A_1A_i$, $A_1M$, and $A_1A_{i+1}$ to find the area, this takes $3$ queries.

It remains to find $A_iA_{i+1}$. Consider the function $f \from \{2, \ldots, 2008\} \to \RR$ that takes $i$ to the answer for query $A_1A_i$. It's a unimodal function, and we're looking for its maximum; we use \href{https://en.wikipedia.org/wiki/Ternary_search}{ternary search} to find it in $\ceil{2\log_{3/2}2007} = 38$ queries. (It's possible to do much better, $\ceil{\log_{\varphi}2007} = 16$ queries, with \href{https://en.wikipedia.org/wiki/Golden-section_search}{golden section search}.)

\item It's impossible when $k$ is odd; we can't distinguish the worlds where everyone tells the truth or everyone lies. Otherwise we need $n$ queries; ask everyone with the same set $A$.

\item It's $10$. Strategy: ask every question, draw the graph with an edge between two vertices if we don't know who is stronger between them. Each vertex has degree at most $1$. (If not, $uv$ and $uw$ are both edges, query $u, v, w$, this eliminates at least one edge.) Split into connected components, each one has at most two, pick a vertex in each component.

Counterstrategy: rank the teams from $1$ to $20$, group them into pairs $(1, 2)$, $(3, 4)$, etc. If a query's teams are all from different pairs, answer whatever. If a query includes two teams from the same pair, name the third team; it's either strongest or weakest. We can't distinguish between the teams in each pair.

\item Answer no unless you have to. In other words, answer no unless it would force the graph to be disconnected. Say the algorithm made $2015$ queries, and there's one edge $uv$ left. By the way we answered, the algorithm must've concluded the graph is connected, even if $uv$ wasn't drawn. Therefore, in the current graph, there must be some path from $u$ to $v$. Let $st$ be the last edge on that path that we answered yes to; WLOG there were paths $u \to s$, $t \to v$. We answered yes because if we answered no, the graph must be disconnected, say because there wouldn't be a path from $p$ to $q$. WLOG there's paths $p\to s$, $q\to t$. But then we could draw $uv$ and find paths $p \to s \to u$, $v \to t \to q$.

\item Two moves. (We can't do it with one move as there are more permutations than partitions.) The idea is if we know $\pi(a) = b$ and $\pi(ac) = bd$, we can deduce $\pi(c) = d$. But we can be more efficient: if we know $\pi(a) = b$ and $\{\pi^{-1}(b), \pi^{-1}(c)\} = \{a, d\}$, we can deduce $\pi^{-1}(c) = d$. And if we know that and $\{\pi(d), \pi(e)\} = \{c, f\}$, we know $\pi(e) = f$, etc.

We do $n$ odd first. For concreteness, we'll do an example with $n = 5$, and label the range with letters. Split $S$ into pairs: $1$, $23$, $45$, and query with $\pi$. Say we get the answer $C$, $AD$, $BE$. We split the answer into pairs the other way: $CA$, $DB$, $E$, and query with $\pi^{-1}$; the above finishes. The $n$ even case is similar: query $1$, $2$, $34$, $56$.

\item Say it takes $f(n)$ queries to solve the problem for $n$. Pretend for a second that we can restart from the beginning. First, let $P$ be the singletons $\{1\}, \{2\}, \cdots, \{n/2\}$. Then use $n$ queries to find how the halves correspond. Now we have two subproblems, with sizes $\floor{n/2}$ and $\ceil{n/2}$. We have to make sure queries from subproblems don't overlap; to do this, add all elements not in the subproblem to every set in $P$. Recursion is $f(n) = f\left(\floor{n/2}\right) + f\left(\ceil{n/2}\right) + n$ and $f(1) = 0$, which solves to $f(n) = n\ceil{\log_2 n}$.

\item It's $2013 - 27 = 1986$. We claim there's a strategy to learn one card's number when we have $28$ cards; we can then induct to learn $1986$ cards. Suppose we have $28$ cards, ask every question. Let $S_k$ be the intersection of all queries with answer $k$. If there's some $|S_k| = 1$ we're done. Otherwise, pick two vertices from each $S_k$ and draw an edge. By construction, there's no independent sets of size $10$. But our graph has $28$ edges and $10$ vertices, so by Tur\'an's theorem the complement has a clique of size $10$, contradiction.

Now we show a counterstrategy for $27$ cards (which solves the problem). Group the vertices into nine directed triangles. By pigeonhole, each query selects two vertices in some triangle, and hence some edge $u \to v$ in that triangle; answer $v$. We can't distinguish between $(a, b, c)$ or $(b, c, a)$ for each of the triangles.

\end{enumerate}

\end{document}
