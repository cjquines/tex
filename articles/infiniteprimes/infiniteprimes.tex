\documentclass[11pt,paper=letter]{scrartcl}
\usepackage[wide]{cjquines}

\begin{document}

\title{Nineteen proofs there are infinitely many primes}
\subtitle{(each in nineteen seconds)}
\author{Carl Joshua Quines}
\date{July 24, 2019}

\maketitle

In all proofs, $p$ is a prime. Many proofs use contradiction, assuming there are finitely many primes $p_1, p_2, \ldots, p_k$.

\begin{enumerate}

\item (Hermite) The smallest prime divisor of $n! + 1$ is greater than $n$.

\item (Euclid) Multiply all the primes and add 1. This number is relatively prime to each prime, contradiction.

\item (Kummer) If the product of all the primes was $N$, then some prime $p$ divides both $N$ and $N-1$, so it also divides $1$, contradiction.

\item (Scimone) Let $N$ be the product of all the primes. But $\sum \frac Np$ is relatively prime to each one.

% \item (Schur) Let $P(x) = x + 1$. Pick $a$ such that $P(a) = b \neq 0$. If $p_1, \ldots, p_k$ are the primes that divide $\cbr{P(n) \mid n \in \ZZ}$. Let $N$ be the product of all these primes. Then $P(bNx) \equiv b \pmod{bN}$. For large enough $x$, $P(bNx) \neq \pm b$, and is relatively prime to each $p_i$, contradiction.

% \item (Braun) Let $N$ be the product of all the primes. As $\sum \frac1p$ is at least $1$, its numerator has a divisor $p$. Then $p$ divides $N\sum \frac1p$, and hence divides $N/p$, contradiction.

\item (Pinasco) Do the sieve of Eratosthenes. The density of integers removed after removing $k$ primes is $(1 - 1/p_1)(1 - 1/p_2)\cdots(1 - 1/p_k)$, by the Chinese remainder theorem. This is less than $1$, so one of the remaining integers is prime.

\item (Goldbach) The numbers $F_n = 2^{2^n} + 1$ satisfy $F_{n+1} - 2 = F_0F_1 \cdots F_n$, so any two are relatively prime. Choose a prime divisor of each $F_n$.

\item (Saidak) Let $a_1 = 2$, and define $a_n = a_{n-1}\del{a_{n-1} + 1}$. As $a_{n-1}$ and $a_{n-1} + 1$ are relatively prime, their product has more prime factors than $a_{n-1}$.

\item (in Engel) The numbers $a_n = 2^{2^{n+1}} + 2^{2^n} + 1$ satisfy $a_n = \del{2^{2^n} - 2^{2^{n-1}} + 1}a_{n-1}$. The two factors are relatively prime, so $a_n$ has at least $n$ prime factors.

\item (Wunderlich) Let $a_n = 2^n - 1$, and observe $\gcd\del{a_m, a_n} = a_{(m, n)}$. So $a_{p_1}, \ldots, a_{p_k}$ are pairwise relatively prime. There are only $k$ primes, so each $a_{p_i}$ has only one prime factor. But $a_{11} = 23 \cdot 89$.

\item (Euler) If the product of all primes was $N$, then $\phi(N) = \prod p_i - 1 \ge 2^{k - 1} \ge 2$. So some integer less than $N$ is relatively prime to $N$, and also each prime.

\item (IMO 1971/3) Consider $2^{(p_1 - 1)(p_2 - 1)\cdots(p_k - 1)} - 3$. By Fermat's little theorem, this is $-2$ mod $p_i$, so it is relatively prime to each prime.

\item (Folklore) Let $p$ be the largest prime. Some prime $q$ divides $2^p - 1$. Then $2^p \equiv 1 \pmod q$, hence $p \mid q - 1$ so $q > p$, contradiction.

\item (Euler) By unique factorization,
$$\prod_p \del{1 - \frac1p}^{-1} = \prod_p \del{1 + \frac1p + \frac1{p^2} + \cdots} = \sum_{n=1}^{\infty}\frac1n,$$
which diverges, so the leftmost product can't be finite.

\item (Euler) Using similar manipulations and a well-known identity, we get $$\prod_p \del{1 - \frac{1}{p^2}}^{-1}= \prod_p \del{1 + \frac1{p^2} + \frac1{p^4} + \cdots} = \sum_{n = 1}^{\infty} \frac{1}{n^2} = \frac{\pi^2}6.$$ But $\pi^2$ is irrational, so the leftmost product can't be finite.

\item (Whang) By de Polignac's formula, $$\nu_p(n!) = \sum_{e=1}^{\infty} \floor{\frac{n}{p^e}} \le \sum_{e=1}^{\infty} \frac n{p^e} = \frac{n}{p-1} \le n \implies \prod_p p^{\nu_p(n!)} \le \prod_p p^n.$$
This is $n!$ on the left and $\del{p_1p_2 \cdots p_k}^n$ on the right. But by Stirling's approximation, $n!$ grows larger than any constant raised to $n$.

\item (Erd\H{o}s) Each positive integer less than $N$ be written as $ab^2$, where $a$ is squarefree and $b^2 < N$. There are $2^k$ choices for $a$, as it's a product of distinct primes, and $\sqrt{N}$ choices for $b$, so $2^k\sqrt{N}$ choices in total. For large $N$, this is less than $N$.

\item (Perott) Let $k$ be the number of primes less than $N$. The number of squarefree integers less than $N$ is at least $$N - \sum_{p \le N} \floor{\frac{N}{p^2}} \ge N\del{1 - \sum_{p \le N} \frac{1}{p^2}} > N\del{1 - \sum_{n=2}^{\infty} \frac{1}{n^2}} > \frac{N}{3}.$$
But there are at most $2^k$ squarefree integers. So $k > \log_2 \frac{N}{3}$.

\item (Thue) Say $n = p_1^{e_1}p_2^{e_2}\cdots p_k^{e_k} < 2^m$ for some $m$. Then $e_1, e_2, \ldots, e_k < m$, so there are at most $m^k$ possibilities for $n$. For large enough $m$, $m^k < 2^m$, contradiction.

\item (Mercer) Let $a + b\ZZ$ be the set of integers congruent to $a$ modulo $b$. Let $N(p) = (1 + p\ZZ) \cup (2 + p\ZZ) \cup \cdots \cup \del{(p-1) + p\ZZ}$. Then
$$\cbr{-1, 1} = N(p_1) \cap N(p_2) \cap \cdots \cap N(p_k),$$
but the intersection of two $a + b \ZZ$s is either empty or is another $a + b \ZZ$ itself.

\end{enumerate}

\subsubsection*{Notes}

\href{https://arxiv.org/pdf/1202.3670.pdf}{Me\u{s}trovi\'c} in his article ``Euclid's theorem on the infinite of primes: A historical survey of the proofs (300 B.C.--2017)'' provides references for each of these proofs, and 163 more of them.

Pinasco's proof counts the number of integers less than $N$ after removing $k$ primes using the principle of inclusion--exclusion; here the proof is adapted as a density argument instead.

The ninth proof is cited in Engel's book \emph{Problem Solving Strategies} to be from a ``recent German contest'', though I couldn't find which one.

IMO 1971 Problem 3 is actually about finding an infinite set of positive integers of the form $2^n - 3$, each pair of which are relatively prime. The problem itself proves there are infinitely many primes, but this adaptation proves it more directly.

I first saw the thirteenth proof from \emph{Proofs from the Book} by Aigner and Ziegler. They don't cite a specific source, citing the result as folklore.

The last proof is Mercer's adaptation of Furstenburg's topological proof, written without mentioning any topological stuff. The original proof was constructing a topology on the integers, where the open sets are $a + b \ZZ$; the conclusion follows from the fact that the intersection of two open sets is an open set.

Thanks to Kevin Chang for inspiring me to do this.

\end{document}
