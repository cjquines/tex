\documentclass[10pt,paper=letter]{scrartcl}
\usepackage[alttitle]{cjquines}

\begin{document}

\title{VCSMS PRIME}
\subtitle{Program for Inducing Mathematical Excellence}
\author{Syllabus}
\date{Carl Joshua Quines}

\maketitle
\setlength{\unitlength}{1in}
\begin{picture}(0,0)
  \put(5.5,0.5){\hbox{\includegraphics[width=0.9in]{logo.png}}}
\end{picture}
\vspace{-2em}

\noindent\textbf{Schedule.} Thursday and Friday, 3:30 PM to 5:30 PM at the math lab.

\noindent\phantom{.}\\\textbf{Homework.} There will be 6 problem sets, due on Wednesdays except the final problem set. These will refer to last year's PRIME problems, which can be found in \url{http://www.cjquines.com/files/prime.pdf}. All problem sets must be submitted completely. The lowest scoring set will be dropped.

\noindent\phantom{.}\\\textbf{Late and incomplete policy.} Late problem sets will be given partial credit. If you have an incomplete problem set, you \emph{must} submit the problems you did not solve. Students who do not answer all problems will be marked as incomplete.

\noindent\phantom{.}\\\textbf{Exams.} There will only be one two-hour final exam, held on session 12, as a mock qualifying stage.

\noindent\phantom{.}\\\textbf{Grading.} Weekly homework is 50\%, the final exam is 30\%, and in-class participation is 20\%.

\noindent\phantom{.}\\\textbf{Calendar.} There will be twelve sessions this year, followed by the PMO Qualifying Stage. Session 11 is after exams, for two hours; session 12 is for the whole day. Grades will be given out on session 12. 

\begin{multicols}{2}

\begin{enumerate}

\item Th, September 14: Angles and areas

\item F, September 15: Functions

\item[W1.] W, September 20: Week 1 homework due

\item Th, September 21: Combinatorial principles

\item F, September 22: Counting and probability

\item[W2.] W, September 27: Week 2 homework due

\item Th, September 28: Trigonometry

\item F, September 29: Circles and polygons

\item[W3.] W, October 4: Week 3 homework due

\item Th, October 5: Sequences and inequalities

\item F, October 6: Algebraic manipulation

\item[W4.] W, October 11: Week 4 homework due

\item Th, October 12: Coordinates

\item F, October 13: Polynomials

\item[W5.] W, October 18: Week 5 homework due

\item[*] W, October 18: GMATIC

\item[*] Th--F, October 19--20: Exams

\item F, October 20: Number theory

\item[*] M--W, October 23--November 1: Sembreak

\item[W6.] F, October 27: Week 6 homework due

\item F, October 27: Metasolving

\item[*] F, October 27: Final exam

\item[*] F, October 27: Grades

\item[*] S, October 28: PMO Qualifying Stage

\end{enumerate}
  
\end{multicols}

\subsubsection*{Reminders}

\begin{itemize}

\item \textbf{Some portion of your grade in Mathematics will be your grade in PRIME.} You are expected to work on homework in lieu of some of your activities in math class. In particular, \emph{you will have enough time for homework}. 

\item You can discuss homework with me or your classmates any time. If you get stuck on a problem, ask.

\item Remember to look at our Facebook group for updates!

\end{itemize}

\end{document}
