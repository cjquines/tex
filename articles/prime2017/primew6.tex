\documentclass[10pt,paper=letter]{scrartcl}
\usepackage[alttitle]{cjquines}

\begin{document}

\title{VCSMS PRIME}
\subtitle{Program for Inducing Mathematical Excellence}
\author{Week 6 Homework}
\date{Due October 27, 2017}

\maketitle
\setlength{\unitlength}{1in}
\begin{picture}(0,0)
  \put(5.5,0.5){\hbox{\includegraphics[width=0.9in]{logo.png}}}
\end{picture}
\vspace{-3.5em}

\subsubsection*{Homework}

Due on Friday, October 27. Along with your usual two sets, you will submit responses to a practice test, instructions below. On Friday, we will have a short session followed by the final exam, then lunch and closing.

\begin{description}
  \item [Set A] (11) \textbf{S3}: Ad hoc 1--2; Factors 1--4; Divisibility 1; Diophantine equations 1--2; Modulo 1--2.
  \item [Set B] (10) \textbf{S3}: Ad hoc 4; Factors 5--7; Divisibility 2, 5; Diophantine equations 3--5; Modulo 3.
  \item [Set C] (11) \textbf{S3}: Ad hoc 3, 6--7; Factors 8, 11; Divisibility 3; Diophantine equations 7--8; Modulo 4--6.
  \item [Set D] (11) \textbf{S3}: Ad hoc 5, 8; Factors 9--10, 12; Divisibility 4, 6; Diophantine equations 6, 9; Modulo 7. \\\textbf{S4}: Ad hoc 9.
\end{description}

\subsubsection*{Practice qualifying stage}

Due on Friday, October 27. The test itself is on our Facebook group.

\begin{itemize}
  \item The test follows the qualifying stage format, which means the time limit is two hours. These should be continuous; do not work for one hour on one day and then one hour on another day.
  \item No aids except scratch paper, ruler, and compass are permitted. No graph paper, protractors, calculators, computers, or mobile phones are allowed. Do not leave the testing area for the duration of the exam.
  \item Write your answers legibly on a clean sheet of short bond paper, with your name.
  \item Do not discuss the problems before Friday, October 27, when they are due.
\end{itemize}

\subsubsection*{Additional problems}

\begin{enumerate}
  \item (HMMT 2004) Given a sequence of six strictly increasing positive integers such that each number (besides the first) is a multiple of the one before it, and the sum of all six numbers is $79$, what is the largest number in the sequence?
  \item (AIME II 2003/2) What is the greatest multiple of $8$ with no two digits the same?
  \item (AIME 1985/7, also MMC) Suppose $a, b, c, d$ are positive integers such that $a^5 = b^4, c^3 = d^2$ and $c-a = 19$. Determine $d-b$.
  \item (Stevens\footnote{Olympiad Number Theory through Challenging Problems, \url{https://aops.com/community/c6h547759}}) Let $f(m)$ be the number of integers $7 < n < m$ such that $\binom{n}{7}$ is divisible by $12$. What is the limit of $f(m)/m$ as $m$ goes to infinity?
  \item (BMO) How many prime numbers are in the sequence $10001, 100010001, 1000100010001, \ldots$?
\end{enumerate}

\subsubsection*{Additional resources}

\begin{itemize}
  \item Previous qualifying stages are on \url{http://pmo.ph/downloads/}. They are also on our Facebook group.
  \item The qualifying stage is roughly post-2000 AMC12 level. The problem statements and coverage are usually more terse and typical of older AIMEs. Past AMC12 and AIME tests are on AoPS at \url{https://aops.com/community/c3415} and \url{https://aops.com/community/c3416}.
\end{itemize}

\end{document}
