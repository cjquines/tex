\documentclass[11pt,paper=letter]{scrartcl}
\usepackage[alttitle]{cjquines}

\begin{document}

\title{Chains and antichains}
\author{CJ Quines}
\date{April 10, 2023}

\maketitle

\subsubsection*{Warmup}

Call two positive integers $a$ and $b$ \textit{friends} if either $a \mid b$ or $b \mid a$, and \textit{enemies} otherwise.
\begin{enumthin}
\item Find $4$ positive integers, such that in every subset of $3$ integers, there is a pair of friends and a pair of enemies.
\item Given $5$ positive integers, prove there's some subset of $3$ integers that are either pairwise friends, or pairwise enemies. (``Pairwise friends'' means every pair is a pair of friends.)
\item What if instead of positive integers we had people? (Assume friendship is symmetric.)
\end{enumthin}

\subsubsection*{Definitions}

A partially ordered set, or \textbf{poset}, is a set $P$ with a binary relation $\le$ that, for all $x, y, z \in P$, is:
\begin{itemthin}
\item reflexive: $x \le x$;
\item antisymmetric: if $x \le y$ and $y \le x$, then $x = y$; and
\item transitive: if $x \le y$ and $y \le z$, then $x \le z$.
\end{itemthin}
We say $x$ and $y$ are \textbf{comparable} if $x \le y$ or $y \le x$, and \textbf{incomparable} otherwise.

\vspace{1em}
\noindent
Poset examples:
\begin{itemthin}
\item $\NN$ with divisibility. Many pairs are incomparable.
\item $\RR$ with $\le$. All pairs are comparable.
\item The power set of $\{1, 2, \ldots, n\}$ with $\subseteq$. In fact, it can be any set of sets with $\subseteq$.
\item People with the ``ancestor of'' relation. In fact, any directed acyclic graph with reachability.
\item $\RR^2$ with $(a, b) \le (c, d)$ if $a \le c$ and $b \le d$. Points to bottom-left or top-right are comparable.
\end{itemthin}
Poset non-examples. Think about why:
\begin{itemthin}
\item $\RR$ with $a \le b$ if $a \le b$ and $|a - b| \le 1$.
\item $\RR^2$ with $(a, b) \le (c, d)$ if $a + b \le c + d$.
\item $\NN$ with $a \le b$ if $\floor{a/2} \le \floor{b/2}$.
\end{itemthin}
\vspace{1em}
\noindent
A subset is a \textbf{chain} if it's pairwise comparable, and an \textbf{antichain} if it's pairwise incomparable.
\begin{itemthin}
\item In $\NN$ with divisibility, the set of $81$'s divisors make a chain. The set of $24$'s divisors don't.
\item In $\RR$ with $\le$, all subsets are chains. All antichains have size $0$ or $1$.
\item In the power set of $\{1, 2, \ldots, n\}$ with $\subseteq$, the subsets with $1$ are neither a chain nor an antichain.
\item In people with the ``ancestor of'' relation, the people in this room (probably) form an antichain.
\item In $\RR^2$ with $\le$, chains are ``increasing'' and antichains are ``decreasing''.
\end{itemthin}
\vspace{1em}
\noindent
\textbf{Remark:} We can also define posets with a binary relation $<$ that is, for all $x, y, z \in P$,
\begin{itemthin}
\item irreflexive: $x \not< x$;
\item asymmetric: if $x < y$, then $y \not< x$; and
\item transitive: if $x < y$ and $y < z$, then $x < z$.
\end{itemthin}
Convince yourself why. Fun fact: you can omit ``irreflexive'' or ``antisymmetric'', but not both.

\subsubsection*{Mirsky, Dilworth, and Erd\H{o}s--Szekeres}

\begin{enumerate}

\item (Mirksy) Let $P$ be a finite poset. The minimum number of antichains whose union is $P$ equals the maximum size of a chain. (This is called the \textit{height} of $P$.)

\textbf{Sketch:} For each element $x$, let $N(x)$ be the maximum size of a chain with largest element $x$. Let $N^{-1}(s)$ be the set of all $x$ such that $N(x) = s$. Each $N^{-1}(s)$ is an antichain. (Why?) The union of $N^{-1}(1), N^{-1}(2), \ldots$ is $P$.

\item (Dilworth) Let $P$ be a finite poset. The minimum number of chains whose union is $P$ equals the maximum size of an antichain. (This is called the \textit{width} of $P$.)

\textbf{Remark:} Proof is omitted because it's not important today. It's harder than Mirsky's. The usual approach is to induct on $|P|$. Feel free to look up the proof later.

\item (Dilworth--Mirsky) In a poset with $ab + 1$ elements, there is a chain of size $a + 1$ or an antichain of size $b + 1$.

\textbf{Sketch:} If there isn't a chain of size $a + 1$, the maximum size of a chain is at most $a$. By Mirsky, the minimum number of antichains whose union is $P$ is at also at most $a$. Make the antichains disjoint by removing elements as needed. There are $ab + 1$ elements in at most $a$ antichains; by pigeonhole some antichain has at least $b + 1$ elements.

\textbf{Remark 1:} There's a similar proof with Dilworth, which is why I called it Dilworth--Mirsky. That's not a real theorem name. You can cite it as Dilworth's theorem too. You can also state the result and say it's well-known; you don't need to know the names.

\textbf{Remark 2:} There are posets with $ab$ elements without chains of size $a + 1$ and antichains of size $b + 1$. For example, consider $b$ chains of size $a$, such that elements from different chains are incomparable. This means Dilworth--Mirsky is sharp, which means we can't make $ab + 1$ smaller. Many of our problems don't talk about sharpness, but I invite you to think about it!

\item (Erd\H{o}s--Szekeres) In a sequence of $ab + 1$ distinct real numbers, there is an increasing subsequence of length $a + 1$ or a decreasing subsequence of length $b + 1$.

\textbf{Sketch:} Consider the poset with $<$ and apply Dilworth--Mirsky.

\textbf{Remark 1:} The usual proof is Seidenberg's, which uses pigeonhole. If you haven't seen it, I recommend looking it up; it's nice.

\textbf{Remark 2:} Many of the solutions today can be written in terms of Erd\H{o}s--Szekeres rather than Dilworth--Mirsky. In fact it's more common to do so, and you'll sometimes read solutions that recast the problem heavily so Erd\H{o}s--Szekeres can be applied. I think this obscures the underlying poset and makes the solution harder to understand.

\end{enumerate}

\subsubsection*{Examples}

\begin{enumerate}

\item (\href{https://mathoverflow.net/questions/421100/is-there-a-configuration-of-5-points-on-the-plane-where-any-two-can-be-covered-b/421101#421101}{\!}) Let $S$ be a set of points on the plane. Suppose that every pair of points in $S$ can be covered by an axis-aligned rectangle that doesn't cover any other point. What is the maximum size of $S$?

\textbf{Sketch:} It's $4$, construction is a tilted square. Consider poset on $S$ with relation $(a, b) \le (c, d)$ if $a \le c$ and $b \le d$. By Dilworth--Mirsky, if $|S| \ge 2 \cdot 2 + 1$, there's monotonic subsequence with $2 + 1$ points, say $A, B, C$. Then covering $A$ and $C$ also covers $B$.

\textbf{Remark:} If you're asked for the minimum or maximum, you also have to show sharpness, typically by giving an example.

\item (Iran 2006) Let $k$ be a positive integer, and let $S$ be a set of open intervals on the real line. Suppose that among any $k + 1$ of these intervals, there are two with a non-empty intersection. Prove that there exists a set of $k$ points on the real line such that each interval in $S$ contains at least one of them.

\textbf{Sketch:} Consider poset on $S$ with relation $(a, b) \le (c, d)$ if $b \le c$. Condition means that chains have size at most $k$. Mirsky means $S$ is union of $k$ antichains. Each antichain has some shared point, choose that.

\item (Erd\H{o}s 1947) If we color the edges of $K_{ab + 1}$ with red and blue, prove there's a red $P_{a + 1}$ or a blue $K_{b + 1}$. (The graph $K_n$ has $n$ vertices, with edges between each pair of vertices. The graph $P_n$ has $n$ vertices, with edges between the $i$th and $(i+1)$st vertices.)

\textbf{Sketch:} Order the vertices. Consider poset with relation $v \le u$ if there's a path from $v$ to $u$ with only red edges and ascending vertices. Chains are red paths. Incomparable vertices are connected by blue edges, so antichains are blue $K_n$s. By Dilworth--Mirsky, either $a + 1$ chain or $b + 1$ antichain.

\item (LMPT 1994) Given $n^5 + 1$ convex polygons, prove that we can choose $n + 1$ that are either pairwise intersecting or pairwise disjoint.

\textbf{Hint:} Let $k$ and $n$ be positive integers. Let $P_1, \ldots, P_{k-1}$ be posets over the same set $S$ with $n^k + 1$ elements. Prove there exists a subset of $S$ with $n + 1$ elements, that's a chain in some $P_i$, or an antichain in all $P_i$s.

\textbf{Sketch:} Define four posets on convex polygons such that two are disjoint iff they're comparable in one of the posets. ($A \le B$ if: $A$ lies entirely below $B$, $A$ lies entirely above $B$, $A$ lies to the bottom-left-ish of $B$, $A$ lies to the top-left-ish of $B$.)

Dilworth--Mirsky on first one, either $n + 1$ chain that's pairwise disjoint, or $n^4 + 1$ antichain. Restrict to $n^4 + 1$, Dilworth--Mirsky on second one, either $n + 1$ chain that's pairwise disjoint, or $n^3 + 1$ antichain, etc. If set is antichain in all posets, it's intersecting.

\end{enumerate}

\subsubsection*{Problems}

\begin{enumerate}

\item (Romania TST 2005) Let $n$ be a positive integer and let $X$ be a set of $n^2+1$ positive integers such that in any subset of $X$ with $n+1$ elements there exist two elements $x\neq y$ such that $x\mid y$. Prove that there exists a subset $\{x_1,x_2,\ldots, x_{n+1} \} \in X$ such that $x_i \mid x_{i+1}$ for all $i=1,2,\ldots, n$.

\item (\href{https://math.stackexchange.com/questions/3689295/how-many-whole-rectangles-can-you-catch-in-a-grid/3704482#3704482}{\!}) Let $R$ be a set of $n^2 + 1$ disjoint rectangles inside a square, whose sides are parallel to the square's sides. Prove that we can partition the square into $n$ rectangles, each of which fully contains a rectangle in $R$. \hint{\ref{h:2}}

\item (MIPT Comb.\ 2019) There are $1001$ rectangles, the $i$th of which has dimensions $\ell_i \times w_i$, where $\ell_i$ and $w_i$ are positive integers such that $|\ell_i - w_i| \le 100$. Prove that there are $21$ rectangles that pairwise fit in each other, rotating by $90\dg$ if necessary. \hint{\ref{h:1}}

\item (\href{https://yufeizhao.com/olympiad/algcomb.pdf}{\!}) Let $G$ be a simple graph. Suppose that it's possible to assign each vertex to one of $\chi$ colors, such that no two adjacent vertices have the same color. Suppose further that we can't do this with $\chi - 1$ colors. Prove that, in this assignment of colors, there is a path of length $\chi$, such that all vertices have different assigned colors. \hint{\ref{h:3}}

\item (Bulgaria 2015) In an olympiad, students received marks in each of four areas: algebra, geometry, number theory and combinatorics. No two students got the same score in any area. A group of students is called \textit{nice} if all students in the group can be ordered in increasing order in two of the four areas. Find the least positive integer $N$ such that among any $N$ students, there exist a nice group of $10$ students. \hint{\ref{h:12}}
% % https://artofproblemsolving.com/community/c6h1846604p12436799

\item (Pach--T\"or\H{o}csik 1994)  Choose $n$ points in the plane, and draw $k^4n + 1$ segments between them. Prove there are $k + 1$ pairwise disjoint segments.

\item (Kalmanson 1973) Let $n$ and $r$ be positive integers. A sequence of points in $\RR^n$ is called \textit{good} if, for each $i = 1, 2, \ldots n$, the $i$th coordinates are either increasing or decreasing. What is the maximum length of a sequence of points in $\RR^n$, such that every good subsequence has length at most $r$? \hint{\ref{h:4}}
\end{enumerate}

\subsubsection*{Harder problems}

\begin{enumerate}[resume]
\item (\href{https://drive.google.com/file/d/1zFLUE59CACJtZKi9FVByllG40dhff1VQ/view}{\!}) Given $500$ distinct points in a unit square, show there's $12$ points $A_1, A_2,\ldots,A_{12}$ such that $A_1A_2 + A_2A_3 + \cdots + A_{11}A_{12} \le 1$. \hint{\ref{h:5}}

\item (USA TST 2008) For a pair $ A = (x_1, y_1)$ and $ B = (x_2, y_2)$ of points on the coordinate plane, let $ d(A,B) = |x_1 - x_2| + |y_1 - y_2|$. We call a pair $ (A,B)$ of (unordered) points harmonic if $ 1 < d(A,B) \leq 2$. Determine the maximum number of harmonic pairs among 100 points in the plane. \hints{\ref{h:6} \ref{h:7}}

\item (K\'arolyi--Rosta 2001) Let $n = 2ab - a - b + 2$. If we color the edges of $K_n$ with red and blue, prove that there's a red $C_{a + 1}$ or a blue $C_{b + 1}$. (The graph $C_n$ is a cycle with $n$ vertices and $n$ edges.) \hint{\ref{h:8}}

\item (Romania TST 2006) Let $m$ and $n$ be positive integers and $S$ be a subset of $ \left\{ 1, 2, \ldots, 2^mn \right\} $ with $ (2^m - 1)n + 1 $ elements. Prove that $S$ contains $m + 1$ distinct integers, $a_0, a_1, \ldots, a_m$ such that $a_{k - 1} \mid a_k$ for all $k = 1, 2, \ldots, m$. \hint{\ref{h:9}}

\item (China TST 2023) In $\triangle ABC$, choose points $P_1, \ldots, P_n$ such that no three are collinear. Prove that we can partition $\triangle ABC$ into $2n + 1$ triangles, such that their vertices are among $A, B, C, P_1, \ldots, P_n$, and at least $n + \sqrt{n} + 1$ of them contain at least one of $A, B, C$. \hint{\ref{h:10}}

\item (ELMO 2016) In a Cartesian coordinate plane, call a rectangle \textit{standard} if all of its sides are parallel to the $x$- and $y$- axes, and call a set of points \textit{nice} if no two of them have the same $x$- or $y$- coordinate. First, Bert chooses a nice set $B$ of $2016$ points in the coordinate plane. To mess with Bert, Ernie then chooses a set $E$ of $n$ points in the coordinate plane such that $B\cup E$ is a nice set with $2016+n$ points. Bert returns and then miraculously notices that there does not exist a standard rectangle that contains at least two points in $B$ and no points in $E$ in its interior. For a given nice set $B$ that Bert chooses, define $f(B)$ as the smallest positive integer $n$ such that Ernie can find a nice set $E$ of size $n$ with the aforementioned properties. Help Bert determine the minimum and maximum possible values of $f(B)$. \hint{\ref{h:11}}
\end{enumerate}

\pagebreak
\subsubsection*{Hints}

\begin{enumerate}
\item \label{h:8} Choose some $p$ and the edges incident to it, do something like the third example for the rest of the graph.
\item \label{h:4} There's a construction where each coordinate is drawn from $\{1, \ldots, r^{2^{n-1}}\}$. Induct on $n$, think about the $n = 2$ example.
\item \label{h:10} Consider $K_{2, \sqrt{n}}$, but connect the $2$ points in the size-$2$ vertex class. Meditate on this graph.
\item \label{h:1} Why can't we have an antichain of size $51$?
\item \label{h:9} Think about $\{1, 2, 4, 8, \ldots\}, \{3, 6, 12, \ldots\}, \{5, 10, 20, \ldots\}, \ldots$.
\item \label{h:6} Out of $3$ increasing or decreasing points, at most $2$ are harmonic.
\item \label{h:11} The maximum is at most $2 \cdot 2016$, because you can add two points in $E$, to the bottom-left and top-right of each existing point. But you can remove some points, like the bottom-leftmost point in $E$.
\item \label{h:3} Follow the third example. Order vertex colors, then consider ascending paths. The antichains contradict minimality of $\chi$.
\item \label{h:7} The graph with harmonic edges has no $K_5$s, T\'uran's finishes.
\item \label{h:5} Use pigeonhole to find a quadrant with $125$ points.
\item \label{h:2} The horizontal projections of rectangles form intervals. Consider a poset on them, like the first example. What can you say about antichains and vertical projections?
\item \label{h:12} The answer is $12^3 + 1$; recall the hint from the fourth example.
\end{enumerate}

\subsubsection*{Sketches}

\begin{enumerate}
\item Consider the $\mid$ poset. Condition means that antichains have size at most $n$. Dilworth--Mirsky means that there's a chain of size at least $n + 1$.

\item The horizontal projections form a poset with $[a, b] < [c, d]$ if $b < c$. By Dilworth--Mirsky, either chain or antichain of size $n + 1$. If chains, draw verical partition; if antichains, draw horizontal partition.

\item WLOG $\ell_i \le w_i$. Consider poset on rectangles with ``fits in'' relation. By Dilworth--Mirsky, either chain of size $20 + 1$ or antichain of size $50 + 1$. Antichain satisfies $\ell_1 < \cdots < \ell_{51} \le w_{51} < \cdots < w_1$ contradicting $|\ell_1 - w_1| \le 100$.

\textbf{Remark:} Compare KMS 2004: Given $1001$ rectangles whose lengths and widths are in $\left\{1, 2, \ldots, 1000\right\},$ prove there exists distinct rectangles $A$, $B$, and $C$, such that $A$ fits in $B$ and $B$ fits in $C$, rotating by $90\dg$ if necessary.

\item Order the colors. Consider poset on vertices with relation $v \le u$ if there's a path from $v$ to $u$ such that vertex colors in path are ascending. If there wasn't a $\chi$ length path, the maximum chain size is at most $\chi - 1$. By Mirsky there's at most $\chi - 1$ antichains that cover $G$. Antichains don't have adjacent vertices, so we can assign them different colors to get a smaller coloring, contradiction.

\item Answer is $9^3 + 1$, give construction. Poset on students with relation $\le$ on both algebra and geometry. By Dilworth--Mirsky, either $9 + 1$ chain and done, or $9^2 + 1$ antichain. In latter case, poset with relation $\le$ on both geometry and number theory. By Dilworth--Mirsky, either $9 + 1$ chain and done, or $9 + 1$ antichain and done.

\textbf{Remark:} Compare K\"ursch\'ak 2020: There are $N$ houses in a city. Every Christmas, Santa visits these $N$ houses in some order. Show that if $N$ is large enough, then it holds that for three consecutive years there are always are $13$ houses such that they have been visited in the same order for two years (out of the three consecutive years). Determine the smallest $N$ for which this holds.
% % https://artofproblemsolving.com/community/c6h2521264p21392594

\item Define four posets on segments such that two segments are disjoint iff they're comparable in one of the posets. Suppose otherwise, each poset has maximum chain size at most $k$. By Dilworth, segments can be partitioned into at most $k$ antichains for each poset. Label segments with $4$-tuples of antichains, have at most $k^4$ possible labels. Segments with same label are in antichains and can't be disjoint, so at most $n$ segments with given label, and at most $k^4n$ segments, contradiction.

\item It's $r^{2^{n-1}}$. Consider poset with $\le$ on first and second coordinates. By Dilworth--Mirsky, if $r^{2^{n-1}} + 1$ points, then either chain or antichain $r^{2^{n-2}} + 1$. Either way first coordinate is monotonic, remove that, repeat. Construction is harder. We'll show there's a construction where each coordinate is drawn from $\{1, \ldots, r^{2^{n-1}}\}$, by induction. Take the construction for $n - 1$. For each $j = 1, \ldots, r^{2^{n-2}}$, we'll make $r^{2^{n-2}}$ copies, where we replace the $j$th coordinate with integers $\{r^{2^{n-1}}(j - 1) + 1, \ldots, r^{2^{n-1}}j\}$ monotonically. The idea is to be similar to the $n = 2$ construction.

\item Divide into quadrants, some quadrant has $11 \cdot 11 + 1$ points by pigeonhole. This forms a poset with $\le$, where $(a, b) \le (c, d)$ if $a \le c$ and $b \le d$. By Dilworth--Mirsky have chain or antichain of size $11 + 1$, the horizontal and vertical projections have lengths that sum to at most $\frac12$.

\item Any $2 \cdot 2 + 1$ points form a poset with $\le$, where $(a, b) \le (c, d)$ if $a \le c$ and $b \le d$. By Dilworth--Mirsky, some $2 + 1$ points $A, B, C$ is increasing or decreasing; either way $d(A, B) + d(B, C) = d(A, C)$, so at most $2$ are harmonic. Consider graph with points as vertices and harmonic pairs as edges; by condition this doesn't contain $K_5$. By Tur\'an, at most $3750$ harmonic pairs. Show achievable.

\item Order the vertices. Consider edges incident to $p$. Either $ab - b + 1$ are red or $ab - a + 1$ are blue, WLOG former. Consider the poset on the $(a - 1)b + 1$ vertices, with $v \le u$ if there's a path from $v$ to $u$ with only red edges and ascending vertices. Chains are red paths, can join to $p$ to make a cycle. Incomparable vertices are connected by blue edges, so antichains are blue $K_n$s. By Dilworth--Mirsky, either $(a - 1) + 1$ chain or $b + 1$ antichain.

\item Consider the $\mid$ poset. Suppose otherwise; then chains have size at most $m$. By Mirsky, $S$ is a union of $m$ antichains. Let $f(a) = \max \left\{ 2^ia \bigmid 2^ia \le 2^mn \right\}$ and $f^{-1}(r) = \left\{ a \in S \bigmid f(a) = r \right\}$. Write $r = 2^{i-1}a$. Then $f^{-1}(r) = S \cap \{a, 2a, \ldots, 2^{i-1}a\}$. Thus $|f^{-1}(r)| \le i$. But it's also $\le m$, because no two elements of $f^{-1}(r)$ can be in the same antichain. Now, $|S| = \sum |f^{-1}(r)|$ over $r$ in the range of $f$. We evaluate this sum as follows: there's $2^{m-1}n$ odd $r$, which each contribute $\le 1$ to the sum; then there's $2^{m-2}n$ even $r$ not divisible by $4$, which each contribute $\le 2$ to the sum, etc. In general, for $i = 1, \ldots, m - 1$, there's $2^{m-1-i}n$ such $r$ that contribute $\le i$ to the sum. In total we get $|S| \le (2^m - 1)n$, contradiction.

\item Assign $(\angle P_iAB, \angle P_iCB)$ to each point. This forms a poset with $\le$, where $(a, b) \le (c, d)$ if $a \le c$ and $b \le d$. Apply Dilworth--Mirsky to get $k = \ceil{\sqrt n}$ points with WLOG increasing $\angle P_iAB$ and $\angle P_iCB$. Draw segments \[
  AP_1, AP_2, \ldots, AP_k, CP_1, CP_2, \ldots, CP_k, BP_1, P_1P_2, \ldots, P_{k-1}P_k,
\]
to partition into $2k + 1$ good triangles, other points add at least one more good triangle.

\item Let $k = 2016$. Minimum is $k - 1$, need to pick points in $E$ between each $x$-coordinate in $B$, construction is points along main diagonal. Maximum is $2k - 2\sqrt{k}$. Consider poset with $\le$ on both coordinates, and longest chain, say length $a$. Draw rectangles between points not in chain to points in chain. This shows we need to add $k - a$ points, one for each point not in longest chain. Similarly, consider longest antichain, say length $b$; we need to add $k - b$ points. In total, $2k - a - b$ points needed. By Dilworth--Mirsky, must have $ab \ge k$, by AM--GM maximum is $2k - 2\sqrt{k}$. Construction is titlted $42 \times 48$ rectangle.
\end{enumerate}

\end{document}
